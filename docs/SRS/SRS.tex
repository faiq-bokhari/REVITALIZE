\documentclass[12pt,letterpaper]{article}
\usepackage[utf8]{inputenc}
\usepackage[margin=1in]{geometry}

\usepackage{booktabs}
\usepackage{tabularx}
\usepackage{float}
\usepackage{hyperref}
\usepackage{indentfirst}
\usepackage{enumerate}
\usepackage[shortlabels]{enumitem}
\usepackage{xcolor}
\usepackage{graphicx}
\usepackage{float}
\usepackage{multirow}
\usepackage{ulem}


\title{System Requirements Specification\\\progname}

\author{\authname}

\date{}

\input{../Comments}
%% Common Parts

\newcommand{\progname}{REVITALIZE} % PUT YOUR PROGRAM NAME HERE
\newcommand{\authname}{Team 13, REVITALIZE
\\ Bill Nguyen and nguyew3
\\ Syed Bokhari and bokhars
\\ Hasan Kibria and kibriah
\\ Youssef Dahab and dahaby
\\ Logan Brown and brownl33
\\ Mahmoud Anklis and anklism} % AUTHOR NAMES                  

\usepackage{hyperref}
    \hypersetup{colorlinks=true, linkcolor=blue, citecolor=blue, filecolor=blue,
                urlcolor=blue, unicode=false}
    \urlstyle{same}
                                


\begin{document}

\maketitle

\begin{table}[hp]
	\caption{Revision History} \label{TblRevisionHistory}
	\begin{tabularx}{\textwidth}{llX}
		\toprule
		\textbf{Date} & \textbf{Developer(s)} & \textbf{Change}\\
		\midrule
		September 29th, 2022 & Youssef Dahab & Project Drivers \\
		October 1st, 2022 & Youssef Dahab & Added Goals of the Project \\
		October 1st, 2022 & Syed Bokhari & Added Functional Requirements and Use Case Diagram \\
		October 2nd, 2022 & Bill Nguyen & Added Non-Functional Requirements and Use Case Diagram \\
		October 3rd, 2022 & Syed Bokhari & Added Work Partitioning Tables \\
		October 4th, 2022 & Youssef Dahab & Added Open issues and New Problems \\
		October 5th, 2022 & Youssef Dahab & Completed Project Issues, My Reflection Appendix, Fixed Spelling Errors\\
		October 5th, 2022 & Mahmoud Anklis & Added Non-Functional Requirements\\
		October 5th, 2022 & Hasan Kibria & Added Constraints and Context Diagram \\
		October 5th, 2022 & Syed Bokhari & Added Dates of Completion and Traceability Matrix \\
		October 5th, 2022 & Logan Brown & Added Activity Diagram\\
		\bottomrule
	\end{tabularx}
\end{table}

\newpage
\tableofcontents
\newpage

\section{Project Drivers}

\subsection{The Purpose of the Project}
Sustaining a healthy lifestyle requires people to keep track of their eating, exercising, and sleeping habits. This can prove to be a daunting and time consuming thing to do especially when most people are very busy with their lives. The purpose of this project to create an all in one health and wellness mobile application that allows users to manage their diet, exercise, and sleep. REVITALIZE is designed to supply users with the means to improve their health by providing them with meal recipe's based on their nutritional preferences, a personalized workouts planner and a sleep tracker. 

\subsection{Scope}
REVITALIZE will allow users to find recipes for meals based on nutritional preferences such as calories per meal, diet selections, allergies to avoid and ingredients included. The user will be able to count their calorie and nutrient intake through the nutrition logger. The workout planner will allow users to choose from an already existing list of workouts or construct their own workout schedule along with weights, sets, and repetitions. The sleep tracker will provide users with information regarding their sleep. There will be a focus on improving user experience throughout the application along with supplying users with accurate data regarding their health.

\subsection{Goals of the Project}
The goal of this project is to make REVITALIZE, for it's stakeholders, the go to, easy to use, quick, and accessible all in one mobile application for effectively and efficiently managing a person's diet, exercise, and sleep to improve their overall health and well being. The goal of making REVITALIZE a mobile application is for it to be easily accessible to users from their phone at any time and place. Users do not have to memorize their health goals or write them down on a piece of paper that they carry with them all the time. The goal of documenting this project is for stakeholders to have a physical system documentation of a functional product that they can refer to when needed. Stakeholders will be able to match the application to the documentation.

\subsection{The Stakeholders}

\subsubsection{Primary Stakeholders}
Adults and teenagers who want to improve and keep track of their overall health and wellness via an easy to use, all in one application.

\subsubsection{Secondary Stakeholders}
Individuals who may not use the application directly for themselves or are not directly involved with the use of the application but have an indirect benefit. For instance, personal trainers can use REVITALIZE to keep track of workouts, sleep, and the overall health of their clients.

\subsubsection{Facilitating Stakeholders}
Team 13 members building the REVITALIZE application along with Dr. Spencer Smith and the 4G06 TAs.

\section{Project Constraints}
\subsection{Mandated Constraints}
\subsubsection{MC 1}
\begin{itemize}
	\item Description: Total project monetary constraints must not exceed \$150. The estimated cost of launching REVITALIZE to the Google Play Store and App Store are \$25 and \$99 respectively.
	\item Rationale: The project must be financially feasible as a no-profit, student-created piece of work
\end{itemize}

\subsubsection{MC 2}
\begin{itemize}
	\item Description: The entire project and its ordered subtasks must be completed as per the deadlines defined by one of the facilitating stakeholders (Dr. Smith)
	\item Rationale: This project is also graded as part of a university course and hence must adhere to strict deadlines
\end{itemize}

\subsection{Naming Conventions and Terminology}
\label{sub:Naming Conventions and Terminology}
\begin{table}[H]
	\caption{Terms and Definitions}
	\centering
	\begin{tabular}{|c|p{10cm}|}
		\hline
		\textbf{Term} & \textbf{Definition}\\
		\hline
		Main Calendar Menu & Main page of application after login page, here one can navigate their way to other menus.\\
		\hline
		Diet Menu & Menu page with functionality naturally inherent to diet-related data\\
		\hline
		Workout Menu & Menu page with functionality naturally inherent to exercise-related data\\
		\hline
		Rest Menu &Menu page with functionality naturally inherent to sleep-related data\\
		\hline
	\end{tabular}
\end{table}


\subsection{Relevant Facts and Assumptions}
\subsubsection{Facts}
\begin{itemize}
	\item The average daily calories intake for an adult is 2000 calories
	\item The average amount of daily sleep required by an adult is 6-8 hours
\end{itemize}

\subsubsection{Assumptions}
\begin{itemize}
	\item User is familiar with basic additive and multiplicative mathematics
	\item User knows how to operate a smart mobile phone
	\item User can read English at a limited working proficiency
\end{itemize}
\section{Context Diagram}
\begin{figure}[H]
	\centering
	\includegraphics[scale=0.5]{cont_diag.png}
	\caption{\textcolor{black} Context diagram for Revitalize}
\end{figure}


\section{Functional Decomposition Diagrams}

\subsection{Work Partitioning}
\begin{table}[h!]
	\caption{Work Partitioning Events}
	\centering
	\begin{tabularx}{\columnwidth}{|X|X|X|X|X|}
		\hline
		\centering\textbf{Event Number} & \centering\textbf{Event Name} & \centering\textbf{Input} & \centering\textbf{Output} & \textbf{Date}\\
		\hline
		1 & \textcolor{red}{\sout{Launch the application login page}}\newline \textcolor{red}{User Login} & \textcolor{red}{\sout{Touch}}\newline \textcolor{red}{Username, Password} & \textcolor{red}{\sout{Main Calender Menu}}\newline \textcolor{red}{Successful Login: Enter Main Menu \newline Unsuccessful Login: Invalid Credentials Message}  & October 31st, 2022\\
		\hline
		2 &  \textcolor{red}{\sout{Opening the signup page}}\newline \textcolor{red}{User Signup} & \textcolor{red}{\sout{Touch}}\newline \textcolor{red}{Username, Password, Email} & \textcolor{red}{\sout{Login Page}}\newline \textcolor{red}{Successful Signup: Enter Login Page \newline Unsuccessful Signup: Invalid Credentials Message} & October 31st, 2022\\
		\hline
		3 & \textcolor{red}{\sout{Opening the main calendar menu}}\newline \textcolor{red}{Main Menu Navigation} & \textcolor{red}{\sout{Touch}}\newline \textcolor{red}{Select Calendar, Diet Section, Exercise Section, Sleep Section} & \textcolor{red}{\sout{ Diet Menu, Workout Menu, Rest Menu}}\newline\textcolor{red}{Navigate to Selected Menu} & October 31st, 2022\\
		\hline
		4 & \textcolor{red}{\sout{Opening the diet menu}}\newline \textcolor{red}{Diet Menu Interaction} & \textcolor{red}{\sout{Touch}}\newline \textcolor{red}{Add, Edit, Delete Food} &\textcolor{red}{\sout{Food List}}\newline \textcolor{red}{Food Log Updated} & November 14th, 2022\\
		\hline
		5 & \textcolor{red}{\sout{Opening the workout menu}}\newline \textcolor{red}{Exercise Menu Interaction} & \textcolor{red}{\sout{Touch}}\newline \textcolor{red}{Add, Edit, Delete Exercise} & \textcolor{red}{\sout{Exercise List}}\newline \textcolor{red}{Exercise Log Updated} & November 14th, 2022\\
		\hline
		6 & \textcolor{red}{\sout{Opening the rest menu}}\newline \textcolor{red}{Sleep Menu Interaction} & \textcolor{red}{\sout{Touch}}\newline \textcolor{red}{Alter Sleep Wheel} & \textcolor{red}{\sout{Sleep log}}\newline \textcolor{red}{Sleep Data Updated} &  November 14th, 2022\\
		\hline
	\end{tabularx}
\end{table}

\begin{table}[h]
	\caption{Work Partitioning Summaries}
	\centering
	\begin{tabular}{|c|p{10cm}|}
		\hline
		\textbf{Event Number} & \textbf{Summary} \\
		\hline
		1 & The user, through the touch input, decides to launch the application.The application launches with the login page and after successful credentials the main calendar menu will be shown. \\
		\hline
		2 & The user, through the touch input, decides to open the sign up page. After successful credentials the login page will be shown. \\
		\hline
		3 & The user, through the touch input, decides to enter the main calendar menu. The user can use touch input to select either the diet menu, workout menu or the rest menu. \\
		\hline
		4 & The user, through the touch input, decides to enter the diet menu. The user can use touch input to view the list of logged food for the calendar day, add custom meals, add recipes and search recipes. The user can also navigate through the calendar for previous date food entries. \\
		\hline
		5 & The user, through the touch input, decides to enter the workout menu. The user can use touch input to view the list of logged exercises for the calendar day, add custom exercises, add preset exercises, and update set and repetition values for each exercise. The user can also navigate through the calendar for previous date workout entries. \\
		\hline
		6 & The user, through the touch input, decides to enter the rest menu. The user can use touch input to alter the sleep data for the current calendar date if inaccurate. The user can also navigate through the calendar for previous date sleep logs. \\
		\hline
	\end{tabular}
\end{table}



\subsection{Use Case Diagram}
\begin{figure}[H]
	\centering
	\includegraphics[width=11cm, height=11cm]{4G06SRSUseCaseDiagram.png}
	\caption{Use case diagram for REVITALIZE}
\end{figure}

\subsection{Activity Diagram}
\begin{figure}[H]
	\centering
	\includegraphics[width=15cm, height=20cm]{4G06SRSActivityDiagram.png}
	\caption{Activity diagram for REVITALIZE}
\end{figure}

\section{Functional Requirements}
\begin{enumerate}[{BE}1.] 
	\item The user launches the application
	\begin{enumerate}[1.]
\item {\textcolor{red}{\sout{The system must display a login page upon the start of the application.}}} \\
{\textcolor{red}{FR1. The application must require users to log in before accessing the main screen.}}
\item {\textcolor{red}{\sout{The login page must display fillable username textbox.}}} \\
{\textcolor{red}{FR2. The system should allow users to log in using a username and password.}}
\item {\textcolor{red}{\sout{The login page must display fillable password textbox.}}} 
\item {\textcolor{red}{\sout{The login page must display a login button.}}} 
\item {\textcolor{red}{\sout{The login page must display a forgot password button.}}} \\
{\textcolor{red}{FR3. The system must allow users to reset their password if they forget it.}}
\item {\textcolor{red}{\sout{The login page must display a stay logged in checkbox.}}} 
\item {\textcolor{red}{\sout{The system must save prior login information if the stayed logged in checkbox is checked.}}} 
\item {\textcolor{red}{\sout{The login page must display a sign up button that redirects to a signup page.}}} \\
{\textcolor{red}{FR4. The system must allow users to sign up for a new account.}}
\item FR5. The system must check the validity of the input parameters in the login page
\item {\textcolor{red}{FR6. The system should display appropriate error messages if there are issues with the login input parameters.}}
\item {\textcolor{red}{FR7. The system should provide appropriate feedback to the user after a successful login}}
	\end{enumerate}
	
	\item The user selects the sign up button
	\begin{enumerate}[resume*]
\item {\textcolor{red}{\sout{The signup page must display fillable username textbox}}}
\\{\textcolor{red}{FR8. The system should allow users to sign up with a unique username, password, and email.}}
\item {\textcolor{red}{\sout{The signup page must display fillable password textbox}}}
\item {\textcolor{red}{\sout{The signup page must display fillable email textbox}}}
\item {\textcolor{red}{\sout{The signup page must display a signup button}}}
\item FR9. The system must check the validity of the input parameters in the signup page
\item {\textcolor{red}{FR10. The system should display appropriate error messages if there are issues with the signup input parameters.}}
\item {\textcolor{red}{FR11. The system should provide appropriate feedback to the user after a successful signup.}}
	\end{enumerate}
	
	\item The user enters the main page after successful login
	\begin{enumerate}[resume*]
\item {\color{red}\sout{The system must display a calendar with the current date on successful login}}
\\{\textcolor{red}{FR12. The system should allow a calendar to be launched to select dates.}}
\item {\color{red}\sout{The system must have a previous day and next day button on each page after successful login}}
\\{\textcolor{red}{FR13. The system should allow the user to navigate to the previous or next day on the calendar.}}
\item {\color{red}\sout{The system must display a back button on each user interface after a section is selected}}
\\{\textcolor{red}{FR14. The system should allow the user to go back to the previous screen or section using a back button.}}
\item {\color{red}\sout{The system must display the sections Diet, Exercise and Rest on the current calendar day.}}
\\{\textcolor{red}{FR15. The system should allow the user to select and view the Diet, Exercise, and Rest sections for the current calendar day.}}
	\end{enumerate}
	
	\item The user enters the Diet section
	\begin{enumerate}[resume*]
\item {\color{red}\sout{The system must prompt the user to height, input dietary, weight, calorie information on initial launch of Diet section}}
\item {\color{red}\sout{The system must save initial user height, dietary, weight, calorie information}}
		\item FR16. The Diet section must initialize with a list of food logged on the current calendar day
\item {\color{red}\sout{The Diet section must display an add food button}}\\
\textcolor{red}{FR17. The Diet section should allow the user to add, edit and delete food entries in the food log}
\item {\color{red}\sout{The Diet section must display a search food button}}\\
\textcolor{red}{FR18. The Diet section should allow the user to search for food}
\item \textcolor{red}{\sout{The search food button must launch a recipe criteria user interface}}\\
\textcolor{red}{FR19. The search food functionality should allow the user to modify criteria and search for recipes}
\item \textcolor{red}{\sout{The recipe criteria user interface must display a list of modifiable criteria and a search button}}
\item FR20. The recipe search must display correct recipe values based on constraints
\item \textcolor{red}{\sout{The recipe search must display a select recipe and add recipe button}}\\
\textcolor{red}{FR21. The recipe search should allow the user to select and add a recipe}
\item \textcolor{red}{\sout{The Diet section must have an add custom meal button}}\\
\textcolor{red}{FR22. The Diet section should allow the user to add a custom meal}
\item \textcolor{red}{\sout{The add custom meal button must have fillable textboxes for recipe information}}\\
\textcolor{red}{FR23. The add custom meal functionality should allow the user to input recipe information}
\item \textcolor{red}{\sout{The previous day and next day button must launch the previous or next calendar entry of the user section.}}
	\end{enumerate}
	
	\item The user enters the Workout section
	\begin{enumerate}[resume*]
		\item  FR24. The Workout section must initialize with a preset list of exercises on the current calendar day
\item {\color{red}\sout{The Workout section must have an add exercise button}}\\
\textcolor{red}{FR25. The Workout section should allow the user to add, edit and delete exercise entries in the exercise log}
\item {\color{red}\sout{The Workout section must have a delete exercise button}}
\item {\color{red}\sout{The exercises must display an edit exercise button that launches the changeable exercise information when clicked}}
		\item  FR26. The Workout section must prompt the user to add repetitions and sets of each exercise logged in the current calendar day
	\end{enumerate}
	
	\item The user enters the Rest section
	\begin{enumerate}[resume*]
		\item  FR27. The Rest section must launch with the sleep statistics of the current calendar day
		\item  FR28. The system must allow the user to alter inaccurate sleep data
	\end{enumerate}
\end{enumerate}

\section{Non-functional Requirements}
\noindent Note: followed the Volere requirements template

\subsection{Look and Feel Requirements}
\subsubsection{Appearance Requirements}
\begin{enumerate}[{LF}1.] 
	\item The application must have a neat and attractive design.\\
	{\textbf{Fit Criterion:} A focus group of primary stakeholders such as teenagers and young adults will look at UI/UX design of application and would require an 85\% approval rating.}
\end{enumerate}

\subsubsection{Style Requirements}
\begin{enumerate}[resume*]  
	\item The application must use colours that are appealing and contrasting to make it more accessible and non-intrusive.\\
	{\textbf{Fit Criterion:} A focus group of primary stakeholders such as teenagers and young adults will test application with a focus on colour and need an 85\% approval rating that the associated colours do not intrude/distract users from overall application.}
\end{enumerate}

\subsection{Usability and Humanity Requirements}
\subsubsection{Ease of Use Requirements}
\begin{enumerate}[{UH}1.] 
	\item All aspects and features of mobile application can be used using only one hand/one finger.\\
	{\textbf{Fit Criterion:} 95\% of stakeholders with varying size hands/fingers are able to use all aspects of mobile application using one hand/one finger.}
\end{enumerate}
\begin{enumerate}[resume*]  
	\item The application home page must be simple so that users can access any feature of the application in under 10 seconds\\
	{\textbf{Fit Criterion:} 90\% of stakeholders can navigate to any application feature from the home page in under 10 seconds. }
	\item The application should be easy to use for targeted demographic\\
	{\textbf{Fit Criterion:} A focus group of primary stakeholders such as teenagers and young adults with youngest age being 14 will test application and need an 85\% approval rating that application was easy to use. }
\end{enumerate}

\subsubsection{Personalization and Internationalization Requirements}
\noindent NOT AVAILABLE

\subsubsection{Learning Requirements}
\begin{enumerate}[resume*] 
	\item Users without any prior experience should be able to use and understand the application within 3 iterations of each feature.\\
	{\textbf{Fit Criterion:} 85\% of stakeholders can use and understand basic/common aspects of all features within 3 iterations.}
\end{enumerate}

\subsubsection{Understandability and Politeness Requirements}
\begin{enumerate}[resume*] 
	\item Associated UI aspects such as buttons, drop-downs, words etc. must be consistent\\
	{\textbf{Fit Criterion:} 85\% of stakeholders agree that all UI aspects are simple, consistent and understandable.}
\end{enumerate}

\subsubsection{Accessibility Requirements}
\begin{enumerate}[resume*] 
	\item Mobile application should be compatible with android screen readers tool for potential users with impaired vision.\\
	{\textbf{Fit Criterion:} Accessibility tests will be conducted and 95\% of application UI should be readable using an android screen reader tool.}
\end{enumerate}

\subsection{Performance Requirements}
\subsubsection{Speed and Latency Requirements}
\begin{enumerate}[{PE}1.] 
	\item All output data of application must take 5 seconds or less to load based on associated inputs.\\
	{\textbf{Fit Criterion:} Developers will run performance tests and ensure all output data loads within 5 seconds or less for 95\% of all API responses and outputs.}
\end{enumerate}

\subsubsection{Safety-Critical Requirements}
\noindent NOT AVAILABLE

\subsubsection{Precision or Accuracy Requirements}
\begin{enumerate}[resume*] 
	\item All output data/numbers should be accurate for double precision floating points.\\
	{\textbf{Fit Criterion:} Perform associated testing (ex. unit testing) to ensure output is accurate for double precision and passes all test cases. }
\end{enumerate}

\subsubsection{Reliability and Availability Requirements}
\begin{enumerate}[resume*] 
	\item Application must have an uptime of 99\%.\\
	{\textbf{Fit Criterion:} Description provides all necessary information. }
\end{enumerate}

\subsubsection{Robustness or Fault-Tolerance Requirements}
\noindent NOT AVAILABLE

\subsubsection{Capacity Requirements}
\begin{enumerate}[resume*] 
	\item Application can be used by a large amounts of users simultaneously.\\
	{\textbf{Fit Criterion:} Application can withstand the usage of at least 50+ users without performance being affected. }
	\item Application can store/save large amount of data.\\
	{\textbf{Fit Criterion:} Application can store/save 1 million+ of data points for all users. }
\end{enumerate}

\subsection{Operational Requirements}
\subsubsection{Expected Physical Environment}
NOT AVAILABLE
\subsubsection{Expected Technological Environment}
\begin{enumerate}[{OE}1.] 
	\item The application is required to connect to the internet.\\
	{\textbf{Fit Criterion:} Users must connect the device that the application is running on to the internet.}
	\item The application is required to run on an Android operating system.\\
	{\textbf{Fit Criterion:} The application runs on Android 11+ operating systems.}
\end{enumerate}
\subsection{Maintainability and Portability Requirements}
\subsubsection{Maintenance Requirements}
\begin{enumerate}[{MP}1.] 
	\item The application's source code must be well commented. \\
	{\textbf{Fit Criterion:} Developers will cross-check each other to confirm well documentation of code. }
	\item The application's source code must maintain the same coding style. \\
	{\textbf{Fit Criterion:} Developers will cross-check each other to confirm that the coding style used follows the \href{https://google.github.io/styleguide/jsguide.html}{\color{blue}Google JS Style Guide}}
\end{enumerate}

\subsubsection{Portability Requirements}
NOT AVAILABLE
\subsubsection{Adaptability Requirements}
\begin{enumerate}[resume*] 
	\item The application must be able to run on Android 11+. \\
	{\textbf{Fit Criterion:} Developers will execute tests on Android 11+ to ensure functionality. }
\end{enumerate}
\subsection{Security Requirements}
\subsubsection{Access Requirements}
\begin{enumerate}[{SE}1.] 
	\item The application must create a secure account for each new sign up registration. \\
	{\textbf{Fit Criterion:} Developers will execute tests to ensure sign-up is fully functional. }
	\item The application cannot have duplicate accounts.  \\
	{\textbf{Fit Criterion:} The application will create only one account per email. }
\end{enumerate}
\subsubsection{Privacy Requirements}
NOT AVAILABLE
\subsubsection{Immunity Requirements}
NOT AVAILABLE
\subsection{Cultural and Political Requirements}
\subsubsection{Cultural Requirements}
\begin{enumerate}[{CU}1] 
	\item The application's interface will be in English. \\
	{\textbf{Fit Criterion:} 100\% of the application's contents will be in English. }
\end{enumerate}
\subsubsection{Political Requirements}
NOT AVAILABLE
\subsection{Legal Requirements}
NOT AVAILABLE

\newpage

\begin{table}[H]
	\begin{center}
		\caption{\textbf{Traceability Matrix for Non-Functional Requirements}}
		\begin{tabularx}{\textwidth}{cc|c|c|c|c|c|c|c|c|c|c|c|c|c|c|}
			\cline{3-13}
			& & \multicolumn{11}{ c|}{Requirements} \\ \cline{3-13}
			& & FR1  & FR2 & FR3 & FR4 & FR5 & FR6 & FR7 & FR8 & FR9 & FR10 & FR11 \\ \cline{1-13}
			\multicolumn{1}{ |c| }{\multirow{10}{*}{Test Cases} } &
			\multicolumn{1}{|c| }{LF1} &X&X&X&X&X&X&&X&&X&X  \\ \cline{2-13}
			\multicolumn{1}{|c| }{} 	                  &
			\multicolumn{1}{|c| }{LF2} &X&X&X&X&X&X&&X&&X&X  \\ \cline{2-13}
			\multicolumn{1}{|c| }{} 	                  &
			\multicolumn{1}{|c| }{UH1} &X&X&X&X&X&X&&X&&X&X \\ \cline{2-13}
			\multicolumn{1}{|c| }{} 	                  &
			\multicolumn{1}{|c| }{UH2} &&&&&&&&&&& \\ \cline{2-13}
			\multicolumn{1}{|c| }{}                        &
			\multicolumn{1}{|c| } {UH3} &X&X&X&X&X&X&&X&&X&X \\ \cline{2-13}
			\multicolumn{1}{|c| }{} 	                  &
			\multicolumn{1}{|c| }{UH4} &X&X&X&X&X&X&&X&&X&X \\ \cline{2-13}
			\multicolumn{1}{|c| }{} 	                  &
			\multicolumn{1}{|c| }{UH5} &X&X&X&X&X&X&&X&&X&X  \\ \cline{2-13}
			\multicolumn{1}{|c| }{}                        &
			\multicolumn{1}{ |c| } {UH6} &X&X&X&X&X&X&&X&&X&X \\ \cline{2-13}
			\multicolumn{1}{|c| }{}                        &
			\multicolumn{1}{ |c| } {PE1} &&&&&&&&&&& \\ \cline{2-13}
			\multicolumn{1}{|c| }{}                        &
			\multicolumn{1}{ |c| }{PE2} &&&&&&&&&&& \\ \cline{2-13}
			\multicolumn{1}{|c| }{}                        &
			\multicolumn{1}{ |c| }{PE3} &&&&&&&&&&& \\ \cline{2-13}
			\multicolumn{1}{|c| }{}                        &
			\multicolumn{1}{ |c| }{PE4} &&&&&&&&&&& \\ \cline{2-13}
			\multicolumn{1}{|c| }{}                        &
			\multicolumn{1}{ |c| }{PE5} &&&&&&&&&&& \\ \cline{2-13}
			\multicolumn{1}{|c| }{}                        &
			\multicolumn{1}{ |c| }{OE1} &&&&&&&X&&X&& \\ \cline{2-13}
			\multicolumn{1}{|c| }{}                        &
			\multicolumn{1}{ |c| }{OE2} &X&&&&&&&&&& \\ \cline{2-13}
			\multicolumn{1}{|c| }{}                        &
			\multicolumn{1}{ |c| }{MP1} &&&&&&&&&&& \\ \cline{2-13}
			\multicolumn{1}{|c| }{}                        &
			\multicolumn{1}{ |c| }{MP2} &&&&&&&&&&& \\ \cline{2-13}
			\multicolumn{1}{|c| }{}                        &
			\multicolumn{1}{ |c| }{MP3} &&&&&&&&&&& \\ \cline{2-13}
			\multicolumn{1}{|c| }{}                        &
			\multicolumn{1}{ |c| }{SE1} &&&&&&&&&&& \\ \cline{2-13}
			\multicolumn{1}{|c| }{}                        &
			\multicolumn{1}{ |c| }{SE2} &&&&&&&&&&& \\ \cline{2-13}
			\multicolumn{1}{|c| }{}                        &
			\multicolumn{1}{ |c| }{CU1} &X&X&X&X&X&X&&X&&X&X \\ \cline{2-13}
			
		\end{tabularx}
	\end{center}
\end{table}

\begin{table}[H]
	\begin{center}
		\caption{\textbf{Traceability Matrix for Non-Functional Requirements Continued}}
		\begin{tabularx}{\textwidth}{cc|c|c|c|c|c|c|c|c|c|c|c|c|c|c|}
			\cline{3-12}
			& & \multicolumn{10}{ c|}{Requirements} \\ \cline{3-12}
			& & FR12  & FR13 & FR14 & FR15 & FR16 & FR17 & FR18 & FR19 & FR20 & FR21 &\\ \cline{1-12}
			\multicolumn{1}{ |c| }{\multirow{10}{*}{Test Cases} } &
			\multicolumn{1}{|c| }{LF1} &X&X&&X&X&X&X&X&X&X  \\ \cline{2-12}
			\multicolumn{1}{|c| }{} 	                  &
			\multicolumn{1}{|c| }{LF2} &X&X&&X&X&X&X&X&X&X  \\ \cline{2-12}
			\multicolumn{1}{|c| }{} 	                  &
			\multicolumn{1}{|c| }{UH1} &X&X&&X&X&X&X&X&X&X \\ \cline{2-12}
			\multicolumn{1}{|c| }{} 	                  &
			\multicolumn{1}{|c| }{UH2} &&&&X&X&X&X&X&X&X \\ \cline{2-12}
			\multicolumn{1}{|c| }{}                        &
			\multicolumn{1}{|c| } {UH3} &X&X&X&X&X&X&X&X&X&X \\ \cline{2-12}
			\multicolumn{1}{|c| }{} 	                  &
			\multicolumn{1}{|c| }{UH4} &X&X&X&X&X&X&X&X&X&X \\ \cline{2-12}
			\multicolumn{1}{|c| }{} 	                  &
			\multicolumn{1}{|c| }{UH5} &X&X&X&X&X&X&X&X&X&X  \\ \cline{2-12}
			\multicolumn{1}{|c| }{}                        &
			\multicolumn{1}{ |c| } {UH6} &X&X&X&X&X&X&X&X&X&X \\ \cline{2-12}
			\multicolumn{1}{|c| }{}                        &
			\multicolumn{1}{ |c| } {PE1} &&&&X&&&X&X&&X \\ \cline{2-12}
			\multicolumn{1}{|c| }{}                        &
			\multicolumn{1}{ |c| }{PE2} &&&&&&&&&X& \\ \cline{2-12}
			\multicolumn{1}{|c| }{}                        &
			\multicolumn{1}{ |c| }{PE3} &&&&&&&&&& \\ \cline{2-12}
			\multicolumn{1}{|c| }{}                        &
			\multicolumn{1}{ |c| }{PE4} &&&&&&&&&& \\ \cline{2-12}
			\multicolumn{1}{|c| }{}                        &
			\multicolumn{1}{ |c| }{PE5} &&&&&&&&&& \\ \cline{2-12}
			\multicolumn{1}{|c| }{}                        &
			\multicolumn{1}{ |c| }{OE1} &&&X&X&&&X&&X&X \\ \cline{2-12}
			\multicolumn{1}{|c| }{}                        &
			\multicolumn{1}{ |c| }{OE2} &&&&&&&&&& \\ \cline{2-12}
			\multicolumn{1}{|c| }{}                        &
			\multicolumn{1}{ |c| }{MP1} &&&&&&&&&& \\ \cline{2-12}
			\multicolumn{1}{|c| }{}                        &
			\multicolumn{1}{ |c| }{MP2} &&&&&&&&&& \\ \cline{2-12}
			\multicolumn{1}{|c| }{}                        &
			\multicolumn{1}{ |c| }{MP3} &&&&&&&&&& \\ \cline{2-12}
			\multicolumn{1}{|c| }{}                        &
			\multicolumn{1}{ |c| }{SE1} &&&X&&&&&&& \\ \cline{2-12}
			\multicolumn{1}{|c| }{}                        &
			\multicolumn{1}{ |c| }{SE2} &&&X&&&&&&& \\ \cline{2-12}
			\multicolumn{1}{|c| }{}                        &
			\multicolumn{1}{ |c| }{CU1} &X&X&&X&X&X&X&X&X&X \\ \cline{2-12}
			
		\end{tabularx}
	\end{center}
\end{table}

\begin{table}[H]
	\begin{center}
		\caption{\textbf{Traceability Matrix for Non-Functional Requirements Continued}}
		\begin{tabularx}{\textwidth}{cc|c|c|c|c|c|c|c|c|c|c|c|c|c|c|}
			\cline{3-12}
			& & \multicolumn{10}{ c|}{Requirements} \\ \cline{3-12}
			& & FR22  & FR23 & FR24 & FR25 & FR26 & FR27 & FR28 & FR29 & FR30 & FR31 &\\ \cline{1-12}
			\multicolumn{1}{ |c| }{\multirow{10}{*}{Test Cases} } &
			\multicolumn{1}{|c| }{LF1} &X&X&X&X&X&X&X&X&X&X  \\ \cline{2-12}
			\multicolumn{1}{|c| }{} 	                  &
			\multicolumn{1}{|c| }{LF2} &X&X&X&X&X&X&X&X&X&X  \\ \cline{2-12}
			\multicolumn{1}{|c| }{} 	                  &
			\multicolumn{1}{|c| }{UH1} &X&X&X&X&X&X&X&X&X&X \\ \cline{2-12}
			\multicolumn{1}{|c| }{} 	                  &
			\multicolumn{1}{|c| }{UH2} &X&X&X&X&X&X&X&X&X&X \\ \cline{2-12}
			\multicolumn{1}{|c| }{}                        &
			\multicolumn{1}{|c| } {UH3} &X&X&X&X&X&X&X&X&X&X \\ \cline{2-12}
			\multicolumn{1}{|c| }{} 	                  &
			\multicolumn{1}{|c| }{UH4} &X&X&X&X&X&X&X&X&X&X \\ \cline{2-12}
			\multicolumn{1}{|c| }{} 	                  &
			\multicolumn{1}{|c| }{UH5} &X&X&X&X&X&X&X&X&X&X  \\ \cline{2-12}
			\multicolumn{1}{|c| }{}                        &
			\multicolumn{1}{ |c| } {UH6}  &X&X&X&X&X&X&X&X&X&X \\ \cline{2-12}
			\multicolumn{1}{|c| }{}                        &
			\multicolumn{1}{ |c| } {PE1} &&&X&&X&&&&X&X \\ \cline{2-12}
			\multicolumn{1}{|c| }{}                        &
			\multicolumn{1}{ |c| }{PE2} &&&&&&&&&& \\ \cline{2-12}
			\multicolumn{1}{|c| }{}                        &
			\multicolumn{1}{ |c| }{PE3} &&&&&&&&&& \\ \cline{2-12}
			\multicolumn{1}{|c| }{}                        &
			\multicolumn{1}{ |c| }{PE4} &&&&&&&&&& \\ \cline{2-12}
			\multicolumn{1}{|c| }{}                        &
			\multicolumn{1}{ |c| }{PE5} &&&&&&&&&& \\ \cline{2-12}
			\multicolumn{1}{|c| }{}                        &
			\multicolumn{1}{ |c| }{OE1} &&&&&X&&&&X&X \\ \cline{2-12}
			\multicolumn{1}{|c| }{}                        &
			\multicolumn{1}{ |c| }{OE2} &&&&&&&&&& \\ \cline{2-12}
			\multicolumn{1}{|c| }{}                        &
			\multicolumn{1}{ |c| }{MP1} &&&&&&&&&& \\ \cline{2-12}
			\multicolumn{1}{|c| }{}                        &
			\multicolumn{1}{ |c| }{MP2} &&&&&&&&&& \\ \cline{2-12}
			\multicolumn{1}{|c| }{}                        &
			\multicolumn{1}{ |c| }{MP3} &&&&&&&&&& \\ \cline{2-12}
			\multicolumn{1}{|c| }{}                        &
			\multicolumn{1}{ |c| }{SE1} &&&&&&&&&& \\ \cline{2-12}
			\multicolumn{1}{|c| }{}                        &
			\multicolumn{1}{ |c| }{SE2} &&&&&&&&&& \\ \cline{2-12}
			\multicolumn{1}{|c| }{}                        &
			\multicolumn{1}{ |c| }{CU1} &X&X&X&X&X&X&X&X&X&X \\ \cline{2-12}
			
		\end{tabularx}
	\end{center}
\end{table}

\begin{table}[H]
	\begin{center}
		\caption{\textbf{Traceability Matrix for Non-Functional Requirements Continued}}
		\begin{tabularx}{\textwidth}{cc|c|c|c|c|c|c|c|c|c|c|c|c|c|c|}
			\cline{3-8}
			& & \multicolumn{6}{ c|}{Requirements} \\ \cline{3-8}
			& & FR32  & FR33 & FR34 & FR35 & FR36 & F37\\ \cline{1-8}
			\multicolumn{1}{ |c| }{\multirow{10}{*}{Test Cases} } &
			\multicolumn{1}{|c| }{LF1} &X&X&X&X&X&X  \\ \cline{2-8}
			\multicolumn{1}{|c| }{} 	                  &
			\multicolumn{1}{|c| }{LF2} &X&X&X&X&X&X  \\ \cline{2-8}
			\multicolumn{1}{|c| }{} 	                  &
			\multicolumn{1}{|c| }{UH1} &X&X&X&X&X&X \\ \cline{2-8}
			\multicolumn{1}{|c| }{} 	                  &
			\multicolumn{1}{|c| }{UH2} &X&X&X&X&X&X \\ \cline{2-8}
			\multicolumn{1}{|c| }{}                        &
			\multicolumn{1}{|c| } {UH3} &X&X&X&X&X&X \\ \cline{2-8}
			\multicolumn{1}{|c| }{} 	                  &
			\multicolumn{1}{|c| }{UH4} &X&X&X&X&X&X \\ \cline{2-8}
			\multicolumn{1}{|c| }{} 	                  &
			\multicolumn{1}{|c| }{UH5} &X&X&X&X&X&X  \\ \cline{2-8}
			\multicolumn{1}{|c| }{}                        &
			\multicolumn{1}{ |c| } {UH6} &X&X&X&X&X&X \\ \cline{2-8}
			\multicolumn{1}{|c| }{}                        &
			\multicolumn{1}{ |c| } {PE1} &&&&&X& \\ \cline{2-8}
			\multicolumn{1}{|c| }{}                        &
			\multicolumn{1}{ |c| }{PE2} &&&&X&X&X \\ \cline{2-8}
			\multicolumn{1}{|c| }{}                        &
			\multicolumn{1}{ |c| }{PE3} &&&&&& \\ \cline{2-8}
			\multicolumn{1}{|c| }{}                        &
			\multicolumn{1}{ |c| }{PE4} &&&&&& \\ \cline{2-8}
			\multicolumn{1}{|c| }{}                        &
			\multicolumn{1}{ |c| }{PE5} &&&&&& \\ \cline{2-8}
			\multicolumn{1}{|c| }{}                        &
			\multicolumn{1}{ |c| }{OE1} &&&&&X&X \\ \cline{2-8}
			\multicolumn{1}{|c| }{}                        &
			\multicolumn{1}{ |c| }{OE2} &&&&&& \\ \cline{2-8}
			\multicolumn{1}{|c| }{}                        &
			\multicolumn{1}{ |c| }{MP1} &&&&&& \\ \cline{2-8}
			\multicolumn{1}{|c| }{}                        &
			\multicolumn{1}{ |c| }{MP2} &&&&&& \\ \cline{2-8}
			\multicolumn{1}{|c| }{}                        &
			\multicolumn{1}{ |c| }{MP3} &&&&&& \\ \cline{2-8}
			\multicolumn{1}{|c| }{}                        &
			\multicolumn{1}{ |c| }{SE1} &&&&&& \\ \cline{2-8}
			\multicolumn{1}{|c| }{}                        &
			\multicolumn{1}{ |c| }{SE2} &&&&&& \\ \cline{2-8}
			\multicolumn{1}{|c| }{}                        &
			\multicolumn{1}{ |c| }{CU1} &X&X&X&X&X&X \\ \cline{2-8}
			
		\end{tabularx}
	\end{center}
\end{table}
\newpage

\section{Project Issues}

\subsection{Open Issues}
The REVITALIZE team members have not completed their investigation of how to make mobile applications compatible with screen reader tools to make REVITALIZE accessible to users with impaired vision.\\

Moreover, the level of difficulty of maintaining and scaling this project is an open issue as it is highly dependent on the implementation. The team has not yet completed it's assessment of the projection of users, costs, and allocated budget to make a current decision on the maintainability and scalability of the project.\\

Another open issue is regarding the way capacity testing will be conducted:
\begin{itemize}
	\item The team has yet to determine how it will find and group a large pool of people simultaneously, ideally 50 or more, to test the application's performance before launch.
	\item Determining how to gather large masses user data points, ideally 1 million+, to test the application's storage capacity and performance.
\end{itemize}

\subsection{Off-the-Shelf Solutions}
There are available products that are similar to the application. An application called MyFitnessPal tracks user goals and calories, breaks down ingredients, and logs activities. Another available mobile application is Samsung Health that tracks various aspects of a user's daily life contributing to their well being such as physical activity, diet, and sleep. It's main dashboard provides access to different features such as dietary monitoring (calories and nutrients absorbed), weight tracking, and sleep monitoring. 

\subsection{New Problems}
Potential new problems that the REVITALIZE team can encounter include:
\begin{itemize}
	\item API failures
	\item Server connection errors
	\item Copy right issues
\end{itemize}
API failures can occur as a result of connectivity issues, a connection breakdown to the URL, or if it's overloaded with requests from the REVITALIZE application. It is possible to encounter copyright issues when using images from the internet to add visuals to the application. This may or may not complicate the graphical user interface design process. A server connection error is less common but may occur due to a network connection error or if the server is offline.

\subsection{Tasks}
Tasks are scheduled and delegated as per the project \href{https://github.com/BillNguyen1999/REVITALIZE/blob/main/projectschedule/REVITALIZE.pdf}{\color{blue}Gantt Chart}. It will be updated throughout the project to include required tasks and their completion status.

\subsection{Migration to the New Product}
NA

\subsection{Risks}
\begin{itemize}
	\item Risk 1: Project may be too complex to complete in eight months because it makes use of multiple built-in and external libraries, other frameworks and application programming interfaces
	\begin{itemize}
		\item Probability of risk becoming a problem: Medium
		\item Contingency plan: Change project scope to meet the minimum number of goals and requirements set by the team to deliver the project
	\end{itemize}
	\item Risk 2: What if primary stakeholders are not too excited about using the REVITALIZE application after it's launched?
	\begin{itemize}
		\item Probability of risk becoming a problem: Medium
		\item Contingency plan: REVITALIZE team members shall speak to primary stakeholders and listen to their feedback to improve the application.
	\end{itemize}
\end{itemize}

\subsection{Costs}
The estimated time cost to deliver this project is eight months. This includes gathering requirements, designing, documenting, implementing, and testing. The cost of launching REVITALIZE to the Google Play Store and App Store are 25\$ and 99\$ respectively.[1]

\subsection{User Documentation and Training}
No user documentation is required to use REVITALIZE. Users will be able to download REVITALIZE as a mobile application from either the Google Play Store or App Store. No training is necessary as using REVITALIZE will be intuitive.

\subsection{Waiting Room}
Ideas that are currently in the waiting room that shall be developed, given enough time, include:
\begin{itemize}
	\item Personal Trainer Integration
	\begin{itemize}
		\item A user's personal trainer can set their diet, workout, and sleep schedules. The trainer can also view user's progress and adjust their schedules.
	\end{itemize}
	\item Sleep Data Predictor
	\begin{itemize}
		\item A sleep data predictor that takes users sleep data to extrapolate their health conditions and give tips on optimizing their sleep.
	\end{itemize}
\end{itemize}

\subsection{Ideas for Solutions}
Possible ideas to make REVITALIZE accessible to all people include designing clear layouts, increasing font sizes, increasing text visibility, using large buttons and controls, and describing user interface elements. Another idea to prevent API failures is to run tests to monitor and track API performance from early development through production.

\section{Reflection Appendix}

\subsection{Knowledge and Skills Needed}

\noindent Bill Nguyen: Skills that the team needs to acquire are end to end development, CI/CD, multiple ways of testing (integration, unit, acceptance tests etc...). This will be needed since project is end to end (front-end, back-end, database etc.), will be using CI/CD to deploy code changes, and will be testing project in multiple ways.\\

\noindent Syed Bokhari: Skills relating to API integration and server communication are needed for the technical development of the application. This is needed because the project depends on the use of multiple APIs for the core functionality of its features. Knowledge of server communication is required due to the nature of application development as the dynamic application information requires constant feedback from the server. \\

\noindent Youssef Dahab: The team will collectively need to acquire documentation skills to successfully complete this project. This includes learning how to write requirements (functional and non functional), how to create use case diagrams, class diagrams, activity diagrams, design documents, how to make a hazard analysis, V\&V plan and report.\\

\noindent Mahmoud Anklis: The team will need to acquire skills pertaining to app development for Android. This will include setting up an emulator that the application can be run and tested on. This will also include setting up the most efficient IDE that will enable the team to develop smoothly.\\ 

\noindent Logan Brown: The team will need to learn how to effectively communicate with stakeholders who do not necessarily understand technical aspects of software. More specifically, this includes knowing how to discern and understand what the client wants and effectively implementing it in the end product. Understanding and responding to stakeholder feedback is another important aspect that must be learned.\\

\noindent Hasan Kibria: Skills that need to be acquired are development with mobile app frameworks and deployment of mobile applications.\\

\subsection{How Knowledge and Skills Will Be Acquired}

\noindent Bill Nguyen: Project will be end to end so will implement code in front-end, back-end etc. and will be integrating all components to make it an end to end project. Will use CI/CD in project to integrate and deploy code, so will add necessary steps in yaml file to make and update CI/CD process. Will research and implement unit, integration, acceptance tests etc. to learn more about multiple types of tests. \\

\noindent Syed Bokhari: The knowledge relating to API integration will be acquired through reading API specific documentation and creating test API applications to test the correct functionality of each API in use. The server communication will be learned through research and development of the project.\\

\noindent Youssef Dahab: An approach to acquire such knowledge is to attend 4G06 course lectures and tutorials, and ask questions to the professor and TAs whenever needed. Another approach is to look up 4G0G lecture notes or past notes from relevant courses such as SFWRENG 2AA4 or 3XA3 and to search or read online on how to properly and correctly document requirements, implementation, test plans, and design documents. I will mainly pursue the second approach as I do not often attend lectures.\\

\noindent Mahmoud Anklis: Online research will play a pivotal role in acquiring the knowledge to develop an Android application. This may include videos, blogs, courses, and tutorials that cover topics from developing to setting up an emulator. The online IDE documentation will also be consulted for any IDE issues. App-Dev books may also be used as a resource.\\

\noindent Logan Brown: Principles drawn from previous courses such as 3RA3, 3I03, as well as lectures in this course will be instrumental in making sure communication goes smoothly. Some trial and error will be required as communication skills are often acquired through experience. Previous experience such as members who have completed co-ops will aid the entire group\\ through teaching what they already know.\\

\noindent Hasan Kibria: Developing/ deploying the codebase that uses a development mobile app framework and using Stack Overflow/ other online resources for ongoing guidance with regards to development as well as deploying.\\

\newpage 

\section{References}
[1] \href{https://www.appypie.com/faqs/how-much-does-it-cost-to-publish-an-app-on-the-app-store}{\color{blue}How much does it cost to publish an app on the app store?}

\end{document}