\documentclass[12pt, titlepage]{article}

\usepackage{booktabs}
\usepackage{tabularx}
\usepackage{hyperref}
\hypersetup{
colorlinks,
citecolor=black,
filecolor=black,
linkcolor=red,
urlcolor=blue
}
\usepackage[round]{natbib}

\usepackage{titlesec}
\usepackage{placeins}
\usepackage{graphicx}
\usepackage{xkeyval}
\usepackage[dvipsnames]{xcolor} % for different colour comments
\usepackage{tabto}
\usepackage{mdframed}
\usepackage{lscape}
\usepackage{multirow}
\usepackage{float}
\newcommand{\comma}{,}
\usepackage[utf8]{inputenc}
\usepackage[margin=1in]{geometry}
\usepackage{booktabs}
\usepackage{tabularx}
\usepackage{hyperref}
\usepackage{float}
\hypersetup{
colorlinks,
citecolor=blue,
filecolor=black,
linkcolor=red,
urlcolor=blue
}
\usepackage[round]{natbib}
\usepackage{indentfirst}
\usepackage{enumerate}
\usepackage[shortlabels]{enumitem}
\usepackage{xcolor}
\usepackage{graphicx}
\usepackage{float}
\usepackage{multirow}
\usepackage{ulem}
\input{../Comments}

\makeatletter
\define@cmdkey      [TP] {test}     {name}       {}
\define@cmdkey      [TP] {test}     {desc}       {}
\define@cmdkey      [TP] {test}     {type}       {}
\define@cmdkey      [TP] {test}     {init}       {}
\define@cmdkey      [TP] {test}     {input}      {}
\define@cmdkey      [TP] {test}     {output}     {}
\define@cmdkey      [TP] {test}     {pass}       {}
\define@cmdkey      [TP] {test}     {user}       {}
\define@cmdkey      [TP] {test}     {result}     {}

\newcounter{TestNum}
\newcommand{\testauto}[1]{
\setkeys[TP]{test}{#1}
\refstepcounter{TestNum}
\begin{mdframed}[linewidth=1pt]
\begin{tabularx}{\textwidth}{@{}p{3cm}X@{}}
	{\bf Test \#\theTestNum:} & {\bf \cmdTP@test@name}\\[\baselineskip]
	{\bf Description:} & \cmdTP@test@desc\\[0.5\baselineskip]
	{\bf Type:} & \cmdTP@test@type\\[0.5\baselineskip]
	{\bf Initial State:} & \cmdTP@test@init\\[0.5\baselineskip]
	{\bf Input:} & \cmdTP@test@input\\[0.5\baselineskip]
	{\bf Output:} & \cmdTP@test@output\\[0.5\baselineskip]
	{\bf Expected:} & \cmdTP@test@pass\\[\baselineskip]
	{\bf Result:} & \cmdTP@test@result
\end{tabularx}
\end{mdframed}
}

\newcommand{\testautob}[1]{
\setkeys[TP]{test}{#1}
\refstepcounter{TestNum}
\begin{mdframed}[linewidth=1pt]
\begin{tabularx}{\textwidth}{@{}p{3cm}X@{}}
	{\bf Test \#\theTestNum:} & {\bf \cmdTP@test@name}\\[\baselineskip]
	{\bf Description:} & \cmdTP@test@desc\\[0.5\baselineskip]
	{\bf Type:} & \cmdTP@test@type\\[0.5\baselineskip]
	{\bf Pass:} & \cmdTP@test@pass\\[\baselineskip]
	{\bf Result:} & \cmdTP@test@result
\end{tabularx}
\end{mdframed}
}

\newcommand{\testmanual}[1]{
\setkeys[TP]{test}{#1}
\refstepcounter{TestNum}
\begin{mdframed}[linewidth=1pt]
\begin{tabularx}{\textwidth}{@{}p{3cm}X@{}}
	{\bf Test \#\theTestNum:} & {\bf \cmdTP@test@name}\\[\baselineskip]
	{\bf Description:} & \cmdTP@test@desc\\[0.5\baselineskip]
	{\bf Type:} & \cmdTP@test@type\\[0.5\baselineskip]
	{\bf Tester(s):} & \cmdTP@test@user\\[0.5\baselineskip]
	{\bf Pass:} & \cmdTP@test@pass\\[\baselineskip]
	{\bf Result:} & \cmdTP@test@result
\end{tabularx}
\end{mdframed}
}

\begin{document}

\title{Verification and Validation Report: REVITALIZE} 
\author{Team 13, 
\\ Bill Nguyen
\\ Syed Bokhari
\\ Hasan Kibria
\\ Mahmoud Anklis
\\ Youssef Dahab
\\ Logan Brown}
\date{\today}

\maketitle

\pagenumbering{roman}

\section{Revision History}

\begin{tabularx}{\textwidth}{p{3cm}p{2cm}X}
\toprule {\bf Date} & {\bf Version} & {\bf Notes}\\
\midrule
March 5th, 2023 & Bill Nguyen & Adding Unit Tests for Workout and Rest Section\\
March 5th, 2023 & Youssef Dahab & Added Functional Requirements Evaluation\\
March 6th, 2023 & Youssef Dahab & Added Changes Due To Testing\\
March 8th, 2023 & Hasan Kibria & Adding Unit Tests for Diet Section\\
March 8th, 2023 & Youssef Dahab & Added Reflection\\
March 8th, 2023 & Logan Brown & Added Non-Functional Requirements Evaluation\\
March 8th, 2023 & Mahmoud Anklis & Added Unit Tests for the User Section and Reflection\\
\bottomrule
\end{tabularx}

~\newpage

\section{Symbols, Abbreviations and Acronyms}

\renewcommand{\arraystretch}{1.2}
\begin{tabular}{l l} 
\toprule		
\textbf{symbol} & \textbf{description}\\
\midrule 
REVITALIZE & Name of application\\
SRS & Software Requirements Specification\\
VnV & Verification and Validation\\
FR & Functional Requirement\\
NFR & Non Functional Requirement\\
LP & Login Page\\
SP & Sign-up Page\\
MP & Main Page or Maintainability and Portability Requirements\\
DS & Diet Section\\
WS & Workout Section\\
RS & Rest Section\\
LF & Look and Feel Requirements\\
UH & Usability and Humanity Requirements\\
PE & Performance Requirement\\
OE & Operational Requirement\\
SE & Security Requirement\\
CU & Cultural Requirement\\
\bottomrule
\end{tabular}\\

\subsection{Symbolic Parameters}

\noindent MINIMUM\_TEST\_SCORE = \hypertarget{MINIMUM_TEST_SCORE}{8.5}\\
MINIMUM\_TEST\_SCORE\_2 = \hypertarget{MINIMUM_TEST_SCORE_2}{9.5}\\
MAXIMUM\_ACCESS\_TIME = \hypertarget{MAXIMUM_ACCESS_TIME}{10}\\
MIN\_APPROVAL\_RATING = \hypertarget{MIN_APPROVAL_RATING}{85\%}\\
MIN\_APPROVAL\_RATING\_2 = \hypertarget{MIN_APPROVAL_RATING_2}{95\%}\\
MIN\_USER\_LOAD = \hypertarget{MIN_USER_LOAD}{50}\\
MIN\_DATA\_POINTS = \hypertarget{MIN_DATA_POINTS}{1000000}

\newpage

\tableofcontents

\listoftables %if appropriate

\listoffigures %if appropriate

\newpage

\pagenumbering{arabic}

This document details the complete testing process for REVITALIZE, as laid out in the project test plan. It contains an evaluation of the project’s functional and non-functional requirements that are defined in the \hyperlink{https://github.com/BillNguyen1999/REVITALIZE/blob/main/docs/SRS/SRS.pdf}{\textbf{SRS}}, the changes made due to testing, and an analysis of the traceability between requirements and modules.

\section{Functional Requirements Evaluation}

\subsection{Login Page}
\testauto{
name = {FR-LP-1},
desc = Testing that login page is displayed upon starting the application,
type = Manual,
init = Loading stage of the login page,
input = An event that loads the login page,
output = Login page is displayed with all necessary components,
pass = ,
result = \textcolor{Green}{PASS}
}
\testauto{
name = {FR-LP-2},
desc = Testing that login page displays fillable username textbox\textcolor{red}{es},
type = Manual,
init = Login page is displayed with username \textcolor{red}{and password} textbox\textcolor{red}{es},
input = \textcolor{red}{\sout{Enter username information in textbox}} \textcolor{red}{Valid username/email and password information entered in their respective textboxes (Valid Inputs: email = "johndoe@gmail.com" and password = "qwerty{}123") },
output =\textcolor{red}{\sout{Username information entered is displayed in textbox}} \textcolor{red}{Returns a success message and user is successfully logged in and redirected to the main page},
pass = ,
result = \textcolor{Green}{PASS}
}
\testauto{
name = {FR-LP-3},
desc =  \textcolor{red}{Verifies that a user fails to log when providing invalid username/email and password information},
type = \textcolor{red}{Automatic, Functional, Dynamic},
init = \textcolor{red}{Login page is displayed with username and password textboxes (Users Current Data: email = "johndoe@gmail.com" and password = "qwerty{}123")},
input = \textcolor{red}{Invalid username/email and password information entered in their respective textboxes (Invalid Input 1: email = "notjohndoe@gmail.com" and password = "qwerty{}123"), (Invalid Input 2: email = "" and password = ""), (Invalid Input 3: email = "johndoe@gmail.com" and password = "wrongqwerty{}123")},
output = \textcolor{red}{User unsuccessfully logins and returns 400 message (Invalid Output 1: "Email not found. Please sign up"), (Invalid Output 2: "Please fill in all fields"), (Invalid Output 3: "Incorrect password")},
pass = ,
result = \textcolor{Green}{PASS}
}
\testauto{
name = {FR-LP-4},
desc = \textcolor{red}{Verifies that the forget password page loads correctly when requested, and that all necessary components are present on the page},
type = Manual\textcolor{red}{, Functional, Dynamic},
init = Login page is displayed with "forgot password" button,
input = \textcolor{red}{\sout{Click forgot password button}} \textcolor{red}{User clicks on the "forgot password" button},
output =  \textcolor{red}{\sout{Display forgot password screen with textbox to enter email}} \textcolor{red}{The system displays the "forgot password" screen with a text box to enter the email.},
pass =  Tester will click on forgot password button and checks if forgot password screen is displayed with textbox to enter email,
result = \textcolor{Green}{PASS}
}
\testauto{
name = {FR-LP-5},
desc = \textcolor{red}{ Verifies the system must allow users to reset their password if they forget it},
type = \textcolor{red}{Manual, Functional, Dynamic},
init = \textcolor{red}{The user is on the login page and has forgotten their password},
input = \textcolor{red}{One string: the user's registered email address},
output =\textcolor{red}{ An email sent to the user's registered email address with a link to reset their password},
pass = ,
result = \textcolor{Green}{PASS}
}
\testauto{
name = {FR-LP-6},
desc =\textcolor{red}{Verifies that the login page displays a sign up button that redirects to the sign up page when clicked},
type = \textcolor{red}{ Manual, Functional, Dynamic},
init =  Login page is displayed with sign up button,
input = Click on the sign up button,
output = \textcolor{red}{\sout{Loads and displays sign up page}} \textcolor{red}{The system redirects the user to the sign up page.},
pass = ,
result = \textcolor{Green}{PASS}
}

\subsection{Signup Page}
\testauto{
name = {FR-SP-1},
desc = Testing that signup page displays fillable username textbox,
type = Manual,
init = \textcolor{red}{\sout{Signup page is displayed with username textbox}} \textcolor{red}{Signup page is displayed with a blank textboxes for username, email, password and confirm password },
input = \textcolor{red}{\sout{Enter username information in textbox}} \textcolor{red}{Enter username, email, password and confirm password information in their respective textboxes. (Valid Inputs: Username = john123, Password = PasSw0rd145, Email = john123@gmail.com, Confirm Password = PasSw0rd145) },
output = \textcolor{red}{\sout{Username information entered is displayed in textbox} } \textcolor{red}{ Upon clicking the "Sign up" button, the user's account is created and the user is logged in to the system. (Valid Output: successful message of "Congrats!','Your account has been successfully created" and user added to database) },
pass = ,
result = \textcolor{Green}{PASS}
}
\testauto{
name = {FR-SP-2},
desc = \textcolor{red}{Testing that a new user can not sign up to REVITALIZE when there is invalid information},
type = \textcolor{red}{\sout{Manual}} \textcolor{red}{Automatic, Functional, Dynamic},
init = \textcolor{red}{\sout{Signup page is displayed with password textbox}} \textcolor{red}{Signup page is displayed with a blank textboxes for username, email, password and confirm password },
input =\textcolor{red}{\sout{Enter password information in textbox}} \textcolor{red}{Enter invalid username, email, password and confirm password information in their respective textboxes. (Invalid Input: Username = !* , Password = abc123, Email = invalid-a-gmail, Confirm Password = abc123)},
output = \textcolor{red}{\sout{Password information entered is displayed in textbox via hidden text}} \textcolor{red}{Upon clicking the "sign up" button, the new user's account is not added to database and alert with failure message will appear. (Invalid Output: failure message of "Invalid Input, Improper Email Address")},
pass = ,
result = \textcolor{Green}{PASS}
}
\testauto{
name = {FR-SP-3},
desc = \textcolor{red}{Tests that a new user can not sign up to REVITALIZE when email already exists in database},
type = \textcolor{red}{Automatic, Functional, Dynamic},
init = \textcolor{red}{Signup page is displayed with a blank textboxes for username, email, password and confirm password. One user already exists in database (Username = john123, Password = PasSw0rd145, Email = john123@gmail.com, Confirm Password = PasSw0rd145)},
input = \textcolor{red}{Enter same username, email, password and confirm password of existing user. (Invalid Input: Username = john123, Password = PasSw0rd145, Email = john123@gmail.com, Confirm Password = PasSw0rd145)},
output = \textcolor{red}{Upon clicking the "sign up" button, the new user's account is not added to database and alert with failure message will appear. (Invalid Output: failure message of "User already exists. Please login")},
pass = ,
result = \textcolor{Green}{PASS}
}
\testauto{
name = {FR-SP-4},
desc = \textcolor{red}{Tests that a new user can not sign up to REVITALIZE when there are empty fields},
type = \textcolor{red}{Automatic, Functional, Dynamic},
init = \textcolor{red}{Signup page is displayed with a blank textboxes for username, email, password and confirm password.},
input = \textcolor{red}{Do not enter username, email, password and confirm password textboxes. (Invalid Input: Username = "" , Password = "", Email = "", Confirm Password = "")},
output = {Upon clicking the "sign up" button, the new user's account is not added to database and alert with failure message will appear. (Invalid Output: failure message of "Please fill in all fields")},
pass = ,
result = \textcolor{Green}{PASS}
}
\testauto{
name = {FR-SP-5},
desc = \textcolor{red}{Tests that a new user can not sign up to REVITALIZE when password and confirm password fields do not match},
type = \textcolor{red}{Automatic, Functional, Dynamic},
init = \textcolor{red}{Signup page is displayed with a blank textboxes for username, email, password and confirm password.},
input = \textcolor{red}{Enter any username and email but enter password and confirm password that do not match. (Invalid Input: Username = "Bob Test" , Password = "qwerty123", Email = "bobtest@gmail.com", Confirm Password = "ytrewq321")},
output = \textcolor{red}{Upon clicking the "sign up" button, the new user's account is not added to database and alert with failure message will appear. (Invalid Output: failure message of "Invalid Input, Passwords do not match")},
pass = ,
result = \textcolor{Green}{PASS}
}

\subsection{Main Page}
\testauto{
name = {FR-MP-1},
desc = Testing that the application displays a calendar with current date on successful login,
type = Manual\textcolor{red}{, Functional, Dynamic},
init = Main page is displayed with calender of current date,
input = An event that loads the main page,
output = Main page is displayed with all necessary components,
pass = ,
result = \textcolor{Green}{PASS}
}
\testauto{
name = {FR-MP-2},
desc = Testing that the application has a previous day and a next day button on each page after successful login,
type = Manual\textcolor{red}{,Functional, Dynamic},
init = Main page and Diet\comma{} Workout\comma{} Rest sections are displayed with previous day and next day buttons,
input = An event that loads the main page\comma{} Diet\comma{} Workout\comma{} Rest sections and the previous day and next day buttons are clicked,
output = Main page\comma{} Diet\comma{} Workout\comma{} Rest sections are displayed with previous day and next day buttons. Once the next day button is clicked\comma{} the calendar refreshes the calendar information for the next day. Once the previous day button is clicked\comma{} the calendar refreshes the calendar information for the previous day,
pass = ,
result = \textcolor{Green}{PASS}
}
\testauto{
name = {FR-MP-3},
desc = Testing that a back button is displayed on each user interface after a section is selected,
type = Manual\textcolor{red}{,Functional, Dynamic},
init = Each interaction after leaving the main page must have a visible back button,
input = An event that loads the next user interface after leaving the main page and the back button is clicked,
output = The next user interface after leaving the main page is displayed with a back button. Once the back button is clicked the main page is loaded,
pass = ,
result = \textcolor{Green}{PASS}
}
\testauto{
name = {FR-MP-4},
desc = Testing that the application displays the sections Diet\comma{} Exercise\comma{} and Rest on the current calendar day,
type = Manual\textcolor{red}{,Functional, Dynamic},
init = Main page is displayed with Diet\comma{} Exercise and Rest buttons available to click,
input = An event that loads the main page and the Diet\comma{} Exercise and Rest buttons are clicked,
output = Main page is displayed with Diet\comma{} Exercise and Rest buttons. If the Diet button is clicked\comma{} the Diet interface is loaded. If the Exercise button is clicked\comma{} the Exercise interface is loaded. If the Rest button is clicked\comma{} the Rest interface is loaded,
pass = ,
result = \textcolor{Green}{PASS}
}

\subsection{Diet Section Page}
\testauto{
name = {FR-DS-1},
desc = \textcolor{red}{Tests that the Diet section must initialize with a list of food logged on the current calendar day},
type = Manual\textcolor{red}{,Functional, Dynamic},
init = \textcolor{red}{\sout{section}} \textcolor{red}{Diet} section is initialized with a list of food logged for the current calender day. \textcolor{red}{Assume this is the current data in the database for test@gmail.com and the date of 2022-10-25: [(foodName = "Oven Fried Chicken II", calories = 200, carbs = 50, fats = 120, protein = 125, foodDate = 2022-10-25 email = "test@gmail.com"),  (foodName = "Lasagna", calories = 100, carbs = 70, fats = 140, protein = 145, foodDate = 2022-10-25 email = "test@gmail.com")]},
input = \textcolor{red}{\sout{An event that loads the rest section}} \textcolor{red}{An event that loads/gets the food log for email = test@gmail.com and foodDate = 2022-10-25},
output = A list of inputted food is loaded for the current calender day \textcolor{red}{. For email = test@gmail.com and foodDate = 2022-10-25, will return a successful message = "Success in getting food log", a list of food names = ["Oven Fried Chicken II", "Lasagna"] and get the total calories, carbs, fats and proteins for the calendar day [calories = 200 + 100, carbs = 50 + 70, fats = 120 + 140, protein = 125 + 145] },
pass = ,
result = \textcolor{Green}{PASS}
}
\testauto{
name = {FR-DS-2},
desc = \textcolor{red}{Verifies that a new meal can be added to database successfully},
type = \textcolor{red}{\sout{Manual}} \textcolor{red}{Automatic, Functional, Dynamic},
init = \textcolor{red}{\sout{Diet section is displayed with add food button}} \textcolor{red}{Database for Diet Section is empty and user (test@gmail.com) wants to add any type of meal (searched recipe or custom meal), for the selected day (2022-10-25)},
input = \textcolor{red}{\sout{Click add food button}} \textcolor{red}{Fill in the required fields for adding a new meal in the database (foodName = "Oven Fried Chicken II", calories = 200, carbs = 50, fats = 120, protein = 125, foodDate = 2022-10-25 email = "test@gmail.com")},
output = \textcolor{red}{\sout{A user interface is launched that lets the user select between searching for food or adding a custom meal}} \textcolor{red}{Input gets added to the database and should have the following success message = "Meal successfully added"},
pass = ,
result = \textcolor{Green}{PASS}
}
\testauto{
name = {FR-DS-3},
desc =  \textcolor{red}{Tests that an existing meal can be updated to database successfully},
type = \textcolor{red}{Automatic, Functional, Dynamic},
init = \textcolor{red}{Database for Diet Section has an existing meal for user (test@gmail.com) with the following data (foodName = "Oven Fried Chicken II", calories = 200, carbs = 50, fats = 120, protein = 125, foodDate = 2022-10-25 email = "test@gmail.com") and user wants to edit calories and protein data},
input = \textcolor{red}{Fill in the fields for the user wants to update in the database (calories = 300, protein = 225, foodDate = 2022-10-25 email = "test@gmail.com")},
output =\textcolor{red}{Input gets updated to the database and should have the following success message = "Success in updating meal" and the updated meal data should look like this (foodName = "Oven Fried Chicken II", calories = 300, carbs = 50, fats = 120, protein = 225, foodDate = 2022-10-25 email = "test@gmail.com")},
pass = ,
result = \textcolor{Green}{PASS}
}
\testauto{
name = {FR-DS-4},
desc =  \textcolor{red}{Verifies that an existing meal can be deleted from database successfully},
type = \textcolor{red}{Automatic, Functional, Dynamic},
init = \textcolor{red}{Database for Diet Section has an existing meal for user (test@gmail.com) with the following data (foodName = "Oven Fried Chicken II" calories = 200, carbs = 50, fats = 120, protein = 125, foodDate = 2022-10-25 email = "test@gmail.com") and user wants to delete this meal data},
input = \textcolor{red}{Fill in the email, food date and food name fields for the user to delete meal data in the database (foodName = "Oven Fried Chicken II", foodDate = 2022-10-25 email = "test@gmail.com" )},
output = \textcolor{red}{Selected meal data is deleted from the database and should have the following success message = "Success in deleting meal"},
pass = ,
result = \textcolor{Green}{PASS}
}
\testauto{
name = {FR-DS-5},
desc =  \textcolor{red}{Tests that searching for a recipe/meal is successful},
type = \textcolor{red}{Manual, Functional, Dynamic},
init = \textcolor{red}{User wants to search for recipe and Diet Recipe Search Screen is loaded, with appropriate textboxes (Name of Meal and Calories) and appropriate dropdowns (List of Diet Types and List of Food Restrictions and Preferences)},
input = \textcolor{red}{Fill all textboxes and dropdowns (foodName = "Chicken", Calories = 1000, Diet Type = "high-protein", Food Restrictions and Preferences = "low-sugar")},
output = \textcolor{red}{Should return a wide range of recipes and meals that satisfies all the conditions from the input},
pass = ,
result = \textcolor{Green}{PASS}
}

\subsection{Workout Section Page}
\testauto{
name = {FR-WS-1},
desc = Testing that the Workout section initializes with a preset list of exercises on the current calendar day,
type = \textcolor{red}{\sout{Manual}} \textcolor{red}{Automatic, Functional, Dynamic},
init = Workout section is initialized with a preset list of exercises of the current calendar day,
input = An event that loads the workout section,
output = A preset list of exercises is loaded for the current calendar day,
pass = ,
result = \textcolor{Green}{PASS}
}
\testauto{
	name = {FR-WS-2},
	desc = \textcolor{red}{Verifies that a new workout can be added to database successfully},
	type = \textcolor{red}{Automatic, Functional, Dynamic},
	init = \textcolor{red}{Database for Workout Section
    is empty and user (test@gmail.com) wants to add new workout, for the selected day (2022-10-25)},
	input = \textcolor{red}{Fill in the required fields for adding a new workout in the
        database (name = "Bicep Curl", weight = 50, sets =
        10, repetitions = 15, dateAdded = 2022-10-25, email = "test@gmail.com")},
	output = \textcolor{red}{Input gets added to the database and should have the
        following success message = ”Success in adding exercise data”},
	pass = ,
	result = \textcolor{Green}{PASS}
}

\testauto{
name = {FR-WS-3},
desc = \textcolor{red}{Tests that an existing workout can be updated to database successfully},
type = \textcolor{red}{Automatic, Functional, Dynamic},
init = \textcolor{red}{Database for Workout Section has an existing workout for user (test@gmail.com) with the following data (name = "Bicep Curl", weight = 50, sets = 10, repetitions = 15, dateAdded = 2022-10-25, email = "test@gmail.com") and user wants to edit sets and repetitions data},
input = \textcolor{red}{ Fill in the fields for the user wants to update in the database (sets = 20, repetitions = 25, dateAdded = 2022-10-25 email = "test@gmail.com")},
output = \textcolor{red}{Input gets updated to the database and should have the following success message = "Success in editing exercise data" and the updated workout data should look like this (name = "Bicep Curl", weight = 50, sets = 20, repetitions = 25, dateAdded = 2022-10-25, email = "test@gmail.com")},
pass = ,
result = \textcolor{Green}{PASS}
}
\testauto{
name = {FR-WS-4},
desc = \textcolor{red}{Tests that an existing workout can be deleted from database successfully},
type = \textcolor{red}{Automatic, Functional, Dynamic},
init = \textcolor{red}{Initial State: Database for Workout Section has an existing workout for user (test@gmail.com) with the following data (name = "Bicep Curl", weight = 50, sets = 10, repetitions = 15, dateAdded = 2022-10-25, email = "test@gmail.com") and user wants to delete this workout data},
input = \textcolor{red}{ Fill in the email, date added and name fields for the user to delete workout data in the database (name = "Bicep Curl", dateAdded = 2022-10-25 email = "test@gmail.com" )},
output =\textcolor{red}{ Selected workout data is deleted from the database and should have the following success message = "Success in deleting exercise data"},
pass = ,
result = \textcolor{Green}{PASS}
}
\testauto{
name = {FR-WS-5},
desc =  \textcolor{red}{Tests that the Workout section must initialize with a list of workouts on the current calendar day},
type = \textcolor{red}{Automatic, Functional, Dynamic},
init = \textcolor{red}{Workout section is initialized with a list of workouts logged for the current calender day. Assume this is the current data in the database for test@gmail.com and the date of 2022-10-25: [((name = "Bicep Curl", weight = 50, sets = 10, repetitions = 15, dateAdded = 2022-10-25, email = "test@gmail.com"),(name = "Bench Press", weight = 50, sets = 13, repetitions = 12, dateAdded = 2022-10-25, email = "test@gmail.com")]},
input = \textcolor{red}{ An event that loads/gets the workout list for email = test@gmail.com and dateAdded = 2022-10-25},
output = \textcolor{red}{ A list of workouts is loaded for the current calender day. For email = test@gmail.com and dateAdded = 2022-10-25, will return a successful message = ”Success in getting exercise list” and will return the following list [((name = "Bicep Curl", weight = 50, sets = 10, repetitions = 15, dateAdded = 2022-10-25, email = "test@gmail.com"),(name = "Bench Press", weight = 50, sets = 13, repetitions = 12, dateAdded = 2022-10-25, email = "test@gmail.com")},
pass = ,
result = \textcolor{Green}{PASS}
}

\subsection{Rest Section Page}
\testauto{
name = {FR-RS-1},
desc = \textcolor{red}{Tests that Rest section launches with sleep statistics of current calendar day},
type = Manual\textcolor{red}{, Functional, Dynamic},
init = Rest section is initialized with the sleep statistics of the current calendar day,
input = An event that loads the rest section,
output = Sleep statistics are loaded for the current calendar day,
pass = ,
result = \textcolor{Green}{PASS}
}
\testauto{
name = {FR-RS-2},
desc = \textcolor{red}{Tests that user can alter inaccurate sleep data},
type = Manual\textcolor{red}{, Functional, Dynamic},
init = Rest section is initialized with the sleep statistics of the current calendar day,
input = Alter sleep data,
output = The sleep data is updated with user changes,
pass = ,
result = \textcolor{Green}{PASS}
}
\testauto{
name = {FR-RS-3},
desc = \textcolor{red}{Tests that a new sleep data can be added to database successfully},
type = \textcolor{red}{Automatic, Functional, Dynamic},
init = \textcolor{red}{Initial State: Database for Rest Section
    is empty and user (test@gmail.com) wants to add new sleep data, for the selected day (2022-10-25)},
input = \textcolor{red}{ Fill in the required fields for adding new sleep data in the
        database (email: "test@gmail.com", sleepHour: 12, bedHour: 10, sleepMinute: 5, bedMinute: 5, dateAdded: 2022-10-25)},
output = \textcolor{red}{ Input gets added to the database and should have the
        following success message = ”Success in adding sleep data”},
pass = ,
result = \textcolor{Green}{PASS}
}
\testauto{
name = {FR-RS-4},
desc =\textcolor{red}{Tests that an existing sleep data can be updated to database successfully},
type = \textcolor{red}{Automatic, Functional, Dynamic},
init =\textcolor{red}{Initial State: Database for Rest Section has an existing sleep data for user (test@gmail.com) with the following data (email: "test@gmail.com", sleepHour: 12, bedHour: 10, sleepMinute: 5, bedMinute: 5, dateAdded: 2022-10-25) and user wants to edit sleep hour and sleep minute data},
input = \textcolor{red}{ Fill in the fields for the user wants to update in the database ( sleepHour: 2, sleepMinute: 10, dateAdded = 2022-10-25 email = "test@gmail.com")},
output = \textcolor{red}{ Input gets updated to the database and should have the following success message = "Success in editing sleep data" and the updated sleep data should look like this (email: "test@gmail.com", sleepHour: 2, bedHour: 10, sleepMinute: 10, bedMinute: 5, dateAdded: 2022-10-25)},
pass = ,
result = \textcolor{Green}{PASS}
}
\testauto{
name = {FR-RS-5},
desc =  \textcolor{red}{Tests that an existing sleep data can be deleted from database successfully},
type = \textcolor{red}{Automatic, Functional, Dynamic},
init = \textcolor{red}{Initial State: Database for Rest Section has an existing sleep data for user (test@gmail.com) with the following data (email: "test@gmail.com", sleepHour: 2, bedHour: 10, sleepMinute: 10, bedMinute: 5, dateAdded: 2022-10-25) and user wants to delete this sleep data},
input = \textcolor{red}{ Fill in the email and date added fields for the user to delete sleep data in the database ( dateAdded = 2022-10-25 email = "test@gmail.com" )},
output = \textcolor{red}{ Selected sleep data is deleted from the database and should have the following success message = "Success in deleting sleep data"},
pass = ,
result = \textcolor{Green}{PASS}
}
\testauto{
name = {FR-RS-6},
desc = \textcolor{red}{Tests that the Rest section must initialize with last saved sleep data on the current calendar day},
type = \textcolor{red}{Automatic, Functional, Dynamic},
init = \textcolor{red}{Rest section is initialized with last saved sleep data for the current calender day. Assume this is the current data in the database for test@gmail.com and the date of 2022-10-25: (email: "test@gmail.com", sleepHour: 2, bedHour: 10, sleepMinute: 10, bedMinute: 5, dateAdded: 2022-10-25)},
input = \textcolor{red}{ An event that loads/gets the last saved sleep data for email = test@gmail.com and dateAdded = 2022-10-25},
output = \textcolor{red}{ The last saved sleep data is loaded for the current calender day. For email = test@gmail.com and dateAdded = 2022-10-25, will return a successful message = ”Success in getting sleep data”},
pass = ,
result = \textcolor{Green}{PASS}
}

\section{Nonfunctional Requirements Evaluation}

\subsection{Look and Feel}

\testmanual{
	name = {NFR-LF1},
	desc = Testing that UI/UX elements are displayed neatly and correctly,
	type = Manual,
	user = Stakeholders,
	pass = Average Q1 survey score of at least \hyperlink{MINIMUM_TEST_SCORE}{MINIMUM\_TEST\_SCORE},
	result = \textcolor{Green}{PASSED} with an agreement of 8.8 out of 10
}

\testmanual{
	name = {NFR-LF2},
	desc = Testing that colours used are acceptable,
	type = Manual,
	user = Stakeholders,
	pass = Average Q2 survey score of at least \hyperlink{MINIMUM_TEST_SCORE}{MINIMUM\_TEST\_SCORE},
	result = \textcolor{Green}{PASSED} with an agreement of 10 out of 10
}

\subsection{Usability and Humanity}

\testmanual{
	name = {NFR-UH1},
	desc = Testing accessibility of application using navigation ability with one finger,
	type = Manual,
	user = Stakeholders,
	pass = Average survey score of at least \hyperlink{MINIMUM_TEST_SCORE_2}{MINIMUM\_TEST\_SCORE\_2},
	result = \textcolor{Green}{PASSED} with an agreement of 10 out of 10
}

\testmanual{
	name = {NFR-UH2},
	desc = Testing navigation speed between screens,
	type = Manual,
	user = Stakeholders,
	pass = Average survey score of at least \hyperlink{MAXIMUM_ACCESS_TIME}{MAXIMUM\_ACCESS\_TIME},
	result = \textcolor{Green}{PASSED} with an agreement of 10 out of 10
}

\testmanual{
	name = {NFR-UH3},
	desc = Testing overall accessibility through average survey result,
	type = Manual,
	user = Stakeholders,
	pass = Average survey score of at least \hyperlink{MINIMUM_TEST_SCORE}{MINIMUM\_TEST\_SCORE},
	result = \textcolor{Green}{PASSED} with an agreement of 9.3
}

\testmanual{
	name = {NFR-UH4},
	desc = Testing learnability of application,
	type = Manual,
	user = Stakeholders,
	pass = \hyperlink{MIN_APPROVAL_RATING}{MIN\_APPROVAL\_RATING} of stakeholders understand functionality in 3 iterations or less,
	result = \textcolor{Green}{PASS}
}

\testmanual{
	name = {NFR-UH5},
	desc = Testing consistency of UI,
	type = Manual,
	user = Stakeholders,
	pass = Average survey score of at least \hyperlink{MINIMUM_TEST_SCORE}{MINIMUM\_TEST\_SCORE},
	result = \textcolor{Green}{PASSED} with an agreement of 9 out of 10
}

\subsection{Performance}

\testmanual{
	name = {NFR-PE1},
	desc = Testing load times of API responses and outputs,
	type = Manual,
	user = Developers,
	pass = Load times below 5 seconds,
	result = \textcolor{Green}{PASS}
}

\testmanual{
	name = {NFR-PE2},
	desc = Testing accuracy of calculated values that contain data/numbers,
	type = Manual,
	user = Developers,
	pass = ,
	result = \textcolor{Green}{PASS}
}

\testmanual{
	name = {NFR-PE3},
	desc = Testing system performance under high load,
	type = Manual,
	user = Developers,
	pass = Previous metrics still pass with \hyperlink{MIN_USER_LOAD}{MIN\_USER\_LOAD} users,
	result = \textcolor{Green}{Tentative Pass}
}

\testmanual{
	name = {NFR-PE4},
	desc = Testing system performance with large amounts of user data,
	type = Manual,
	user = Developers,
	pass = Previous metrics still pass with \hyperlink{MIN_DATA_POINTS}{MIN\_DATA\_POINTS} per user,
	result = \textcolor{Green}{PASS}
}

\subsection{Operational}

\testmanual{
	name = {NFR-OE1},
	desc = Testing if all features are loaded with stable internet connection,
	type = Manual,
	user = Developers,
	pass = ,
	result = \textcolor{Green}{PASS}
}

\subsection{Maintainability and Portability}

\testmanual{
	name = {NFR-MP1},
	desc = Testing maintainability through cross referencing developer comments,
	type = Manual,
	user = Developers,
	pass = ,
	result = \textcolor{Green}{PASS}
}

\subsection{Security}

\testmanual{
	name = {NFR-SE1},
	desc = Testing that passwords are hashed and user data is secure,
	type = Manual,
	user = Developers,
	pass = ,
	result = \textcolor{Green}{PASS}
}

\testmanual{
	name = {NFR-SE2},
	desc = Testing that emails can only have 1 associated account,
	type = Manual,
	user = Developers,
	pass = ,
	result = \textcolor{Green}{PASS}
}

\subsection{Cultural and Political}

\testmanual{
	name = {NFR-CU1},
	desc = Testing that the displayed language is in English,
	type = Manual,
	user = Developers,
	pass = ,
	result = \textcolor{Green}{PASS}
}

\textcolor{red}{
\subsection{Overview of Testers}
NFS tests were conducted by Logan, Faiq, and Youssef
}



\section{Comparison to Existing Implementation}	

\noindent N/A

\section{Unit Testing}

\subsection{Workout Section}
\newpage

Unit tests for the workout section: \url{https://github.com/BillNguyen1999/REVITALIZE/blob/main/src/SERVER/backend/test/exercise.test.js}.
\begin{table}[h]
\centering
\small
\begin{tabularx}{\textwidth}{|X|X|p{3cm}|p{2.5cm}|p{2.5cm}|X|}
	\hline
	Test ID & FR & Inputs & Expected Values & Actual Values & Result \\
	\hline
	WS1 & FR-WS-1 and FR-WS-5 & \{email: 'test@gmail.com', dateAdded: '2022-01-01'\}  & [{ name: 'Exercise 1' }, { name: 'Exercise 2' }] & [{ name: 'Exercise 1' }, { name: 'Exercise 2' }] & \textcolor{Green}{PASS} \\
	\hline
	WS2 & FR-WS-1 and FR-WS-5 & \{email: 'fail@gmail.com', 'dateAdded: 2022-01-01'\}  & 'Error in getting exercise list' & 'Error in getting exercise list' & \textcolor{Green}{PASS} \\
	\hline
	WS3 & FR-WS-2 & \multicolumn{3}{p{8cm}|}{\centering \{ \textcolor{Green}{success: true, message: 'Success in adding exercise data'}, id: 'exerciseid', email: 'test@gmail.com', name: 'Test Exercise', sets: 3, repetitions: 10, weight: 50, dateAdded: '2022-03-07'\}}  &  \textcolor{Green}{PASS}\\
	\hline
	WS4 & FR-WS-3 & \{email: 'test@gmail.com', dateAdded: '2022-01-01', name:'push-ups'\}  & \{success: true, message: 'Success in deleting exercise data'\} & \{success: true, message: 'Success in deleting exercise data'\} & \textcolor{Green}{PASS} \\
	\hline
	WS5 & FR-WS-3 & \{email: 'notfound@gmail.com', dateAdded: '2022-01-01', name:'push-ups'\}  & \{success: false, message: 'Was not able to delete selected exercise data'\} & \{success: false, message: 'Was not able to delete selected exercise data'\} & \textcolor{Green}{PASS} \\
	\hline
\end{tabularx}
\caption{Workout Section Unit Tests Part 1}
\label{table:workout-unit-tests}
\end{table}

\newpage

\begin{table}[h]
\centering
\small
\begin{tabularx}{\textwidth}{|X|X|p{3cm}|p{2.5cm}|p{2.5cm}|X|}
	\hline
	Test ID & FR & Inputs & Expected Values & Actual Values & Result \\
	\hline
	WS6 & FR-WS-4 & params: \{ email: 'example@gmail.com', dateAdded: '2022-01-01', name: 'exerciseName' \},
	body: \{ reps: 10, sets: 3 \}  & \{success: true, message: 'Success in editing exercise data'\} & \{success: true, message: 'Success in editing exercise data'\} & \textcolor{Green}{PASS} \\
	\hline
	WS7 & FR-WS-4 & params: \{ email: 'notfound@gmail.com', dateAdded: '2022-03-07', name: 'push-ups' \},
	body: \{ sets: 3, reps: 10\}  & \{success: false, message: "Was not able to find appropriate exercise data to edit" \} & \{success: false, message: "Was not able to find appropriate exercise data to edit" \} & \textcolor{Green}{PASS} \\
	\hline
	WS8 & FR-WS-1 and FR-WS-5 & \{email: test@gmail.com, name:'pushup', dateAdded: 2022-01-01\}  & \{success: true,
	message: 'Success in getting exercise data' \} & \{success: true, message: 'Success in getting exercise data' \} & \textcolor{Green}{PASS} \\
	\hline
	WS9 & FR-WS-1 and FR-WS-5 & \{email: 'test@gmail.com', name: 'pushup', dateAdded: 'invalid-date' \} & \{success: false, message: 'Error in getting exercise data' \} & \{success: false, message: 'Error in getting exercise data' \} & \textcolor{Green}{PASS} \\
	\hline
\end{tabularx}
\caption{Workout Section Unit Tests Part 2}
\label{table:workout-unit-tests2}
\end{table}

\newpage

\subsection{Rest Section}

Unit tests for the rest section: \url{https://github.com/BillNguyen1999/REVITALIZE/blob/main/src/SERVER/backend/test/sleep.test.js}.

\begin{table}[h]
\centering
\small
\begin{tabularx}{\textwidth}{|X|X|p{3cm}|p{2.5cm}|p{2.5cm}|X|}
	\hline
	Test ID & FR & Inputs & Expected Values & Actual Values & Result \\
	\hline
	RS1 & FR-RS-1 and FR-RS-2 & \{email: 'test@gmail.com', dateAdded: '2022-01-01'\}  & \{success: true, message: 'Success in getting sleep data'\} & \{success: true, message: 'Success in getting sleep data'\} & \textcolor{Green}{PASS} \\
	\hline
	RS2 & FR-RS-1 and FR-RS-2 & \{email: 'test@gmail.com', dateAdded: 'invalid-date'\}  & \{success: false, message: 'Error in getting sleep data'\} & \{success: false, message: 'Error in getting sleep data'\} & \textcolor{Green}{PASS} \\
	\hline
	RS3 & FR-RS-1 & \multicolumn{3}{p{8cm}|}{\centering \{ \textcolor{Green}{success: true, message: 'Success in adding sleep data'}, id: 'sleepid', email: 'test@gmail.com', sleepHour: 12, bedHour: 10, sleepMinute: 5, bedMinute: 5, dateAdded: '2022-03-07'\}}  &  \textcolor{Green}{PASS}\\
	\hline
	RS4 & FR-RS-2 & \{email: 'test@gmail.com', dateAdded: '2022-01-01'\}  & \{success: true, message: 'Success in deleting sleep data'\} & \{success: true, message: 'Success in deleting sleep data'\} & \textcolor{Green}{PASS} \\
	\hline
	RS5 & FR-RS-2 & \{email: 'notfound@gmail.com', dateAdded: '2022-01-01'\}  & \{success: false, message: 'Was not able to delete selected sleep data'\} & \{success: false, message: 'Was not able to delete selected sleep data'\} & \textcolor{Green}{PASS} \\
	\hline
\end{tabularx}
\caption{Rest Section Unit Tests Part 1}
\label{table:rest-unit-tests}
\end{table}

\newpage

\begin{table}[h]
\centering
\small
\begin{tabularx}{\textwidth}{|X|X|p{3cm}|p{2.5cm}|p{2.5cm}|X|}
	\hline
	Test ID & FR & Inputs & Expected Values & Actual Values & Result \\
	\hline
	RS6 & FR-RS-2 & params:\{ email: 'example@gmail.com', dateAdded: '2022-01-01'\},
	body: \{ sleepHour: 12, bedHour: 11, sleepMinute: 57, bedMinute: 47\}  & \{success: true, message: 'Success in editing sleep data'\} & \{success: true, message: 'Success in editing sleep data'\} & \textcolor{Green}{PASS} \\
	\hline
	RS7 & FR-RS-2 & params: \{ email: 'notfound@gmail.com', dateAdded: '2022-03-07'\},
	body: \{ sleepHour: 12, bedHour: 11, sleepMinute: 57, bedMinute: 47\}  & \{success: false, message: "Was not able to find appropriate sleep data to edit" \} & \{success: false, message: "Was not able to find appropriate sleep data to edit" \} & \textcolor{Green}{PASS} \\
	\hline
\end{tabularx}
\caption{Rest Section Unit Tests Part 2}
\label{table:rest-unit-tests2}
\end{table}

\newpage

\subsection{Diet Section}

Unit tests for the rest section: \url{https://github.com/BillNguyen1999/REVITALIZE/blob/main/src/SERVER/backend/test/foodLog.test.js}.

\begin{table}[h]
\centering
\small
\begin{tabularx}{\textwidth}{|X|X|p{3cm}|p{2.5cm}|p{2.5cm}|X|}
	\hline
	Test ID & FR & Some Inputs & Some Expected Values & Corresponding Actual Values & Result \\
	\hline
	DS1 & FR-DS-3 & \{email: 'test@gmail.com', foodDate: 2022-03-08\}  & \{success: true, message: 'Success in getting food log'\} & \{success: true, message: 'Success in getting food log'\} & \textcolor{Green}{PASS} \\
	\hline
	DS2 & FR-DS-2 & \{email: 'test@gmail.com', foodDate: 2022-03-08, calories: 1\}  & \{success: true, message: 'Meal successfully added', calories: 1\} & \{success: true, message: 'Meal successfully added', calories: 1\} & \textcolor{Green}{PASS} \\
	\hline
	DS3 & FR-DS-3 & \{email: 'test@gmail.com', foodDate: 2022-03-08, foodName: 'name'\}  & \{success: true, message: 'Success in deleting meal'\} & \{success: true, message: 'Success in deleting meal''\} & \textcolor{Green}{PASS} \\
	\hline
	DS4 & FR-RS-8 & \{email: 'test@gmail.com', foodDate: 2022-03-08\}  & \{success: true, message: 'Success in updating meal'\} & \{success: true, message: 'Success in updating meal'\} & \textcolor{Green}{PASS} \\
	\hline
\end{tabularx}
\caption{Diet Section Unit Tests Part 1}
\label{table:diet-unit-tests}
\end{table}

\newpage

\subsection{User Section}

Unit tests for the User section: \url{https://github.com/BillNguyen1999/REVITALIZE/blob/main/src/SERVER/backend/test/user.test.js}.
\begin{table}[h]
\centering
\small
\begin{tabularx}{\textwidth}{|X|X|p{3cm}|p{2.5cm}|p{2.5cm}|X|}
	\hline
	Test ID & FR & Inputs & Expected Values & Actual Values & Result \\
	\hline
	US1 & FR-SP-1, FR-SP-2, FR-SP-3 and FR-SP-5  & \{name:'Test Name',email: 'test123@gmail.com', password: '12345'\}  & Status Code = 201 & Status Code = 201 & \textcolor{Green}{PASS} \\
	\hline
	
\end{tabularx}
\caption{User Section Unit Test}
\label{table:workout-unit-tests}
\end{table}

\newpage

\section{Changes Due to Testing}
Formal testing did not reveal any necessary changes in terms of module interfacing, decomposition, or internal design. Changes made to code were to address bugs and logical errors revealed by the testing plan. User interface improvements were made throughout the development process in response to feedback from developers and informal testers.

\section{Automated Testing}
Jest was used to automate the unit tests
\section{Trace to Requirements}
\begin{table}[H]
	\begin{center}
		\caption{\textbf{Traceability Matrix for Login Page Functional Requirements}}
		\begin{tabularx}{\textwidth}{cc|c|c|c|c|c|c|c|c|c|c|}
			\cline{3-9}
			& & \multicolumn{7}{ c|}{Requirements} \\ \cline{3-9}
			& & FR1  & FR2 & FR3 & FR4 & FR5 & FR6 & FR7 \\ \cline{1-9}
			\multicolumn{1}{ |c| }{\multirow{6}{*}{Test Cases} } &
			\multicolumn{1}{|c| }{ FR-LP-1} &X&&&&&& \\ \cline{2-9}
			\multicolumn{1}{|c| }{} 	                  &
			\multicolumn{1}{|c| }{ FR-LP-2} &&X&&&&&X \\ \cline{2-9}
			\multicolumn{1}{|c| }{} 	                  &
			\multicolumn{1}{|c| }{ FR-LP-3} &&&X&&&& \\ \cline{2-9}
			\multicolumn{1}{|c| }{} 	                  &
			\multicolumn{1}{|c| }{ FR-LP-4} &&&&X&&& \\ \cline{2-9}
			\multicolumn{1}{|c| }{}                        &
			\multicolumn{1}{|c| }{ FR-LP-5} &&&&&X&& \\ \cline{2-9}
			\multicolumn{1}{|c| }{} 	                  &
			\multicolumn{1}{|c| }{ FR-LP-6} &&&&&&X& \\ \cline{1-9}
		\end{tabularx}
	\end{center}
\end{table}

\begin{table}[H]
	\begin{center}
		\caption{\textbf{Traceability Matrix for Signup Page Functional Requirements}}
		\begin{tabularx}{\textwidth}{cc|c|c|c|c|c|c|c|}
			\cline{3-6}
			& & \multicolumn{4}{ c|}{Requirements} \\ \cline{3-6}
			& & FR\textcolor{red}{8}  & FR\textcolor{red}{9} & FR\textcolor{red}{10} & FR\textcolor{red}{11} \\ \cline{1-6}
			\multicolumn{1}{ |c| }{\multirow{5}{*}{Test Cases} } &
			\multicolumn{1}{|c| }{ FR-SP-1} &X&&&X \\ \cline{2-6}
			\multicolumn{1}{|c| }{} 	                  &
			\multicolumn{1}{|c| }{ FR-SP-2} &&X&X& \\ \cline{2-6}
			\multicolumn{1}{|c| }{} 	                  &
			\multicolumn{1}{|c| }{ FR-SP-3} &&X&X& \\ \cline{2-6}
			\multicolumn{1}{|c| }{} 	                  &
			\multicolumn{1}{|c| }{ FR-SP-4} &&X&X& \\ \cline{2-6}
            \multicolumn{1}{|c| }{} 	                  &
			\multicolumn{1}{|c| }{ FR-SP-5} &&X&X& \\ \cline{1-6}
		\end{tabularx}
	\end{center}
\end{table}

\begin{table}[H]
	\begin{center}
		\caption{\textbf{Traceability Matrix for Main Page Functional Requirements}}
		\begin{tabularx}{\textwidth}{cc|c|c|c|c|c|c|}
			\cline{3-6}
			& & \multicolumn{4}{ c|}{Requirements} \\ \cline{3-6}
			& & FR\textcolor{red}{12}  & FR\textcolor{red}{13} & FR\textcolor{red}{14} & FR\textcolor{red}{15} \\ \cline{1-6}
			\multicolumn{1}{ |c| }{\multirow{4}{*}{Test Cases} } &
			\multicolumn{1}{|c| }{ FR-MP-1} &X&&& \\ \cline{2-6}
			\multicolumn{1}{|c| }{} 	                  &
			\multicolumn{1}{|c| }{ FR-MP-2} &&X&& \\ \cline{2-6}
			\multicolumn{1}{|c| }{} 	                  &
			\multicolumn{1}{|c| }{ FR-MP-3} &&&X& \\ \cline{2-6}
			\multicolumn{1}{|c| }{} 	                  &
			\multicolumn{1}{|c| }{ FR-MP-4} &&&&X \\ \cline{1-6}
		\end{tabularx}
	\end{center}
\end{table}

\begin{table}[H]
	\begin{center}
		\caption{\textbf{Traceability Matrix for Diet Page Functional Requirements}}
		\begin{tabularx}{\textwidth}{cc|c|c|c|c|c|c|c|c|c}
			\cline{3-10}
			& & \multicolumn{8}{ c|}{Requirements} \\ \cline{3-10}
			& & FR\textcolor{red}{16}  & FR\textcolor{red}{17}  & FR\textcolor{red}{18}  & FR\textcolor{red}{19}  & FR\textcolor{red}{20}   & FR\textcolor{red}{21} & FR\textcolor{red}{22} & FR\textcolor{red}{23} \\ \cline{1-10}
			\multicolumn{1}{ |c| }{\multirow{5}{*}{Test Cases} } &
			\multicolumn{1}{|c| }{ FR-DS-1} &X&&&&&&&& \\ \cline{2-10}
			\multicolumn{1}{|c| }{} 	                  &
			\multicolumn{1}{|c| }{ FR-DS-2} &&X&&&&&&& \\ \cline{2-10}
			\multicolumn{1}{|c| }{} 	                  &
			\multicolumn{1}{|c| }{ FR-DS-3} &&X&&&&&&& \\ \cline{2-10}
			\multicolumn{1}{|c| }{} 	                  &
			\multicolumn{1}{|c| }{ FR-DS-4} &&X&&&&&&& \\ \cline{2-10}
			\multicolumn{1}{|c| }{} 	                  &
			\multicolumn{1}{|c| }{ FR-DS-5} &&&X&X&X&X&X&X& \\ \cline{1-10}
		\end{tabularx}
	\end{center}
\end{table}

\begin{table}[H]
	\begin{center}
		\caption{\textbf{Traceability Matrix for Workout Page Functional Requirements}}
		\begin{tabularx}{\textwidth}{cc|c|c|c|c|c|c|}
			\cline{3-5}
			& & \multicolumn{3}{ c|}{Requirements} \\ \cline{3-5}
			& & FR\textcolor{red}{24}  & FR\textcolor{red}{25} & FR\textcolor{red}{26} \\ \cline{1-5}
			\multicolumn{1}{ |c| }{\multirow{6}{*}{Test Cases} } &
			\multicolumn{1}{|c| }{ FR-WP-1} &X&& \\ \cline{2-5}
			\multicolumn{1}{|c| }{} 	                  &
			\multicolumn{1}{|c| }{ FR-WP-2} &&X& \\ \cline{2-5}
			\multicolumn{1}{|c| }{} 	                  &
			\multicolumn{1}{|c| }{ FR-WP-3} &&X& \\ \cline{2-5}
			\multicolumn{1}{|c| }{} 	                  &
			\multicolumn{1}{|c| }{ FR-WP-4} &&X& \\ \cline{2-5}
			\multicolumn{1}{|c| }{}                        &
			\multicolumn{1}{|c| }{ FR-WP-5} &X&&X \\ \cline{1-5}
		\end{tabularx}
	\end{center}
\end{table}

\begin{table}[H]
	\begin{center}
		\caption{\textbf{Traceability Matrix for Rest Section Functional Requirements}}
		\begin{tabularx}{\textwidth}{cc|c|c|c|c|c|}
			\cline{3-4}
			& & \multicolumn{2}{ c|}{Requirements} \\ \cline{3-4}
			& & FR\textcolor{red}{27}  & FR\textcolor{red}{28} \\ \cline{1-4}
			\multicolumn{1}{ |c| }{\multirow{6}{*}{Test Cases} } &
			\multicolumn{1}{|c| }{ FR-RS-1} &X& \\ \cline{2-4}
			\multicolumn{1}{|c| }{}                        &
			\multicolumn{1}{|c| }{ FR-RS-2} &&X \\ \cline{2-4}
            \multicolumn{1}{|c| }{}                        &
			\multicolumn{1}{|c| }{ \textcolor{red}{FR-RS-3}} &&X \\ \cline{2-4}
            \multicolumn{1}{|c| }{}                        &
			\multicolumn{1}{|c| }{ \textcolor{red}{FR-RS-4}} &&X \\ \cline{2-4}
            \multicolumn{1}{|c| }{}                        &
			\multicolumn{1}{|c| }{ \textcolor{red}{FR-RS-5}} &&X \\ \cline{2-4}
            \multicolumn{1}{|c| }{}                        &
			\multicolumn{1}{|c| }{ \textcolor{red}{FR-RS-6}} &&X \\ \cline{1-4}
		\end{tabularx}
	\end{center}
\end{table}

\begin{table}[H]
	\begin{center}
		\caption{\textbf{Traceability Matrix for Look and Feel Nonfunctional Requirements}}
		\begin{tabularx}{\textwidth}{cc|c|c|c|c|c|}
			\cline{3-4}
			& & \multicolumn{2}{ c|}{Requirements} \\ \cline{3-4}
			& & LF1  & LF2 \\ \cline{1-4}
			\multicolumn{1}{ |c| }{\multirow{3}{*}{Test Cases} } &
			\multicolumn{1}{|c| }{ NFR-LF1} &X& \\ \cline{2-4}
			\multicolumn{1}{|c| }{}                        &
			\multicolumn{1}{|c| }{ NFR-LF22} &&X \\ \cline{1-4}
		\end{tabularx}
	\end{center}
\end{table}


\begin{table}[H]
	\begin{center}
		\caption{\textbf{Traceability Matrix for Usability and Humanity Nonfunctional Requirements}}
		\begin{tabularx}{\textwidth}{cc|c|c|c|c|c|c|c|c|c|}
			\cline{3-8}
			& & \multicolumn{6}{ c|}{Requirements} \\ \cline{3-8}
			& & UH1  & UH2 & UH3 & UH4 & UH5 & UH6 \\ \cline{1-8}
			\multicolumn{1}{ |c| }{\multirow{6}{*}{Test Cases} } &
			\multicolumn{1}{|c| }{ NFR-UH1} &X&&&&& \\ \cline{2-8}
			\multicolumn{1}{|c| }{} 	                  &
			\multicolumn{1}{|c| }{ NFR-UH2} &&X&&&& \\ \cline{2-8}
			\multicolumn{1}{|c| }{} 	                  &
			\multicolumn{1}{|c| }{ NFR-UH3} &&&X&&& \\ \cline{2-8}
			\multicolumn{1}{|c| }{} 	                  &
			\multicolumn{1}{|c| }{ NFR-UH4} &&&&X&&\\ \cline{2-8}
			\multicolumn{1}{|c| }{}                        &
			\multicolumn{1}{|c| }{NFR-UH5} &&&&&X& \\ \cline{1-8}
		\end{tabularx}
	\end{center}
\end{table}

\begin{table}[H]
	\begin{center}
		\caption{\textbf{Traceability Matrix for Perfromance Nonfunctional Requirements}}
		\begin{tabularx}{\textwidth}{cc|c|c|c|c|c|c|c|c|}
			\cline{3-7}
			& & \multicolumn{5}{ c|}{Requirements} \\ \cline{3-7}
			& & PE1  & PE2 & PE3 & PE4 & PE5 \\ \cline{1-7}
			\multicolumn{1}{ |c| }{\multirow{5}{*}{Test Cases} } &
			\multicolumn{1}{|c| }{ NFR-PE1} &X&&&& \\ \cline{2-7}
			\multicolumn{1}{|c| }{} 	                  &
			\multicolumn{1}{|c| }{ NFR-PE2} &&X&&& \\ \cline{2-7}
			\multicolumn{1}{|c| }{} 	                  &
			\multicolumn{1}{|c| }{ NFR-PE3} &&&X&& \\ \cline{2-7}
			\multicolumn{1}{|c| }{} 	                  &
			\multicolumn{1}{|c| }{ NFR-PE4} &&&&X& \\ \cline{2-7}
            \multicolumn{1}{|c| }{} 	                  &
			\multicolumn{1}{|c| }{ \textcolor{red}{NFR-PE5}} &&&&&X \\ \cline{1-7}
		\end{tabularx}
	\end{center}
\end{table}

\begin{table}[H]
	\begin{center}
		\caption{\textbf{Traceability Matrix for Operational Nonfunctional Requirements}}
		\begin{tabularx}{\textwidth}{cc|c|c|c|c|c|}
			\cline{3-4}
			& & \multicolumn{2}{ c|}{Requirements} \\ \cline{3-4}
			& & OE1  & OE2 \\ \cline{1-4}
			\multicolumn{1}{ |c| }{\multirow{2}{*}{Test Cases} } &
			\multicolumn{1}{|c| }{ NFR-OE1} &X& \\ \cline{2-4}
            \multicolumn{1}{|c| }{} 	                  &
			\multicolumn{1}{|c| }{ \textcolor{red}{NFR-OE2}} &&X \\ \cline{1-4}
		\end{tabularx}
	\end{center}
\end{table}

\begin{table}[H]
	\begin{center}
		\caption{\textbf{Traceability Matrix for Maintainability and Portability Nonfunctional Requirements}}
		\begin{tabularx}{\textwidth}{cc|c|c|c|c|c|c|}
			\cline{3-5}
			& & \multicolumn{3}{ c|}{Requirements} \\ \cline{3-5}
			& & MP1  & MP2 & MP3 \\ \cline{1-5}
			\multicolumn{1}{ |c| }{\multirow{2}{*}{Test Cases} } &
			\multicolumn{1}{|c| }{ NFR-MP1} &X&& \\ \cline{2-5}
			\multicolumn{1}{|c| }{}                        &
			\multicolumn{1}{|c| }{ NFR-MP2} &&X& \\ \cline{1-5}
		\end{tabularx}
	\end{center}
\end{table}

\begin{table}[H]
	\begin{center}
		\caption{\textbf{Traceability Matrix for Security Nonfunctional Requirements}}
		\begin{tabularx}{\textwidth}{cc|c|c|c|c|c|}
			\cline{3-4}
			& & \multicolumn{2}{ c|}{Requirements} \\ \cline{3-4}
			& & SE1  & SE2 \\ \cline{1-4}
			\multicolumn{1}{ |c| }{\multirow{3}{*}{Test Cases} } &
			\multicolumn{1}{|c| }{NFR-SE1} &X& \\ \cline{2-4}
			\multicolumn{1}{|c| }{}                        &
			\multicolumn{1}{|c| }{NFR-SE2} &&X \\ \cline{1-4}
		\end{tabularx}
	\end{center}
\end{table}

\begin{table}[H]
	\begin{center}
		\caption{\textbf{Traceability Matrix for Cultural and Political Nonfunctional Requirements}}
		\begin{tabularx}{\textwidth}{cc|c|c|c|c|}
			\cline{3-3}
			& & \multicolumn{1}{ c|}{Requirements} \\ \cline{3-3}
			& & CU1 \\ \cline{1-3}
			\multicolumn{1}{ |c| }{\multirow{1}{*}{Test Cases} } &
			\multicolumn{1}{|c| }{NFR-CU1} &X \\ \cline{1-3}
		\end{tabularx}
	\end{center}
\end{table}

\section{Trace to Modules}		
\begin{table}[H]
\centering
\small
\renewcommand{\arraystretch}{0.9}
\begin{tabular}{p{0.2\textwidth} p{0.6\textwidth}}
	\toprule
	\textbf{Req.} & \textbf{Modules}\\
	\midrule
	FR-LP-1 & M3\\
	FR-LP-2 & M3\\
	FR-LP-3 & M3\\
	FR-LP-4 & M3\\
	FR-LP-5 & M3\\
	FR-LP-6 & M3\\
	FR-LP-7 & M3\\
	FR-LP-8 & M3\\
	FR-LP-9 & M3\\
	FR-SP-1 & M18\\
	FR-SP-2 & M18\\
	FR-SP-3 & M18\\
	FR-SP-4 & M18\\
	FR-SP-5 & M18\\
	FR-MP-1 & M1\\
	FR-MP-2 & M1\\
	FR-MP-3 & M1\\
	FR-MP-4 & M1\\
	FR-DS-1 & M7\\
	FR-DS-2 & M7\\
	FR-DS-3 & M7\\
	FR-DS-4 & M8\\
	FR-DS-5 & M8, M10\\
	FR-DS-6 & M11\\
	FR-DS-7 & M11\\
	FR-DS-8 & M9\\
	FR-WP-1 & M14\\
	FR-WP-2 & M14\\
	FR-WP-3 & M15\\
	FR-WP-4 & M15\\
	FR-WP-5 & M17\\
	FR-RS-1 & M5\\
	FR-RS-2 & M6\\
	\bottomrule
\end{tabular}
\caption{Trace Between Requirements and Modules}
\label{TblRT}
\end{table}
\section{Code Coverage Metrics}
N/A
\section{Reflection Appendix}
Bill Nguyen: for the vnv plan, it was more formulation rather than implementation, we looked at how we were going to test our project rather than actually doing it. For the vnv report it was more the implementation of our formulation where we wrote actual unit/automated tests and tested our project fully and than compared it to our vnv plan to see what requirements etc. did we satisfy and maybe find things we need to improve on.\\\\
Hasan Kibria: In comparison to the vnv plan, the vnv report was more based on practicality an implementation. To complete it fully, there was real code and test cases that had to be thought of an implemented so that they could then be documented in the vnv report. In the vnv plan it was more of an outlook of what we envisioned our testing to look like.\\\\
Syed Bokhari: The VNV plan focuses on formulating the testing approach and strategies, while the VNV report is more concerned with the implementation and documentation of the actual testing process. The VNV report involves the creation and execution of test cases, which are then compared to the plan to identify any gaps or areas for improvement. The VNV plan provides a high-level view of the testing process, while the VNV report is a more detailed account of the actual testing activities.\\\\
Youssef Dahab: Both the VnV plan and VnV report take inspiration from the functional and non-functional requirements in the SRS document. The VnV plan described how we were going to test our functional and non-functional requirements while the VnV report described the results of performing those tests.\\\\
Logan Brown: The VnV plan was more abstract without knowledge of the implementation. The VnV report documents the more refined and directed tests that were performed which could now be completed due to the implementation being more concrete. I have a better idea of how VnV is carried out and the importance of "faking the design process" in the initial stages to make later VnV much easier.\\\\
Mahmoud Anklis: The VnV plan is designed to come up with a testing and verification methodology that would ensure that the software application adheres to the functional and non-functional requirements. On the other hand, the VnV report focuses on the actual execution of the tests which requires implementation steps.\\\\

\bibliographystyle{plainnat}

\bibliography{../../refs/References}

\end{document}