\documentclass{article}

\usepackage{booktabs}
\usepackage{tabularx}
\usepackage{hyperref}
\usepackage{graphicx}
\usepackage{pdflscape}

\hypersetup{
colorlinks=true,       % false: boxed links; true: colored links
linkcolor=red,          % color of internal links (change box color with linkbordercolor)
citecolor=green,        % color of links to bibliography
filecolor=magenta,      % color of file links
urlcolor=cyan           % color of external links
}

\title{Hazard Analysis\\\progname}

\author{\authname}

\date{}

\input{../Comments}
%% Common Parts

\newcommand{\progname}{REVITALIZE} % PUT YOUR PROGRAM NAME HERE
\newcommand{\authname}{Team 13, REVITALIZE
\\ Bill Nguyen and nguyew3
\\ Syed Bokhari and bokhars
\\ Hasan Kibria and kibriah
\\ Youssef Dahab and dahaby
\\ Logan Brown and brownl33
\\ Mahmoud Anklis and anklism} % AUTHOR NAMES                  

\usepackage{hyperref}
    \hypersetup{colorlinks=true, linkcolor=blue, citecolor=blue, filecolor=blue,
                urlcolor=blue, unicode=false}
    \urlstyle{same}
                                


\begin{document}

\maketitle
\thispagestyle{empty}

~\newpage

\pagenumbering{roman}

\begin{table}[hp]
	\caption{Revision History} \label{TblRevisionHistory}
	\begin{tabularx}{\textwidth}{llX}
		\toprule
		\textbf{Date} & \textbf{Developer(s)} & \textbf{Change}\\
		\midrule
		October 15th, 2022 & Bill Nguyen & FMEA \\
		October 19th, 2022 & Youssef Dahab & Introduction, Purpose, and Scope \\
		\bottomrule
	\end{tabularx}
\end{table}

~\newpage

\tableofcontents

~\newpage

\pagenumbering{arabic}

\section{Introduction}
This document is a hazard analysis of REVITALIZE.

\subsection{Hazard Definition}
As per Nancy Leveson's work, a hazard is a property or condition in a system together with a condition in the environment that has the potential to cause harm or damage. In REVITALIZE, there are safety (keeping records) and security (restricting access to data) hazards.

\subsection{Other Used Terms in this Document}

\subsubsection{Failure}
A failure in the REVITALIZE mobile application could occur when there is a deviation between the actual and expected output. Furthermore, a failure could also occur when a certain state causes REVITALIZE, or a component of REVITALIZE (login, database, etc.), to fail and therefore not achieve its necessary function.

\subsubsection{Safety}
REVITALIZE's safety is its freedom from harm. However, it's not an absolute. Safety is a global property of REVITALIZE. Therefore, when two or more REVITALIZE components interact together, the resulting emergent behaviour may or may not be safe.

\section{Purpose of Hazard Analysis and Scope}
Hence, the purpose of this document is to identify hazards pertaining to the REVITALIZE mobile application, and their causes, and then specify ways to eliminate them or mitigate their effect. The team cannot “bolt on” safety after implementing the application. Therefore, by identifying hazards and developing safety requirements before implementing, REVITALIZE team members will be able to react when their application fails to be safe. This becomes beneficial when something “bad” happens. The team will have a recovery plan.
\\\\ The scope of this project is to create an all in one health and wellness mobile application that allows users to manage their diet, exercise, and sleep by providing them with meal recipe’s based on their nutritional preferences, a personalized workouts planner and a sleep tracker.

\section{System Boundaries and Components}

\section{Critical Assumptions}

\wss{These assumptions that are made about the software or system.  You should
	minimize the number of assumptions that remove potential hazards.  For instance,
	you could assume a part will never fail, but it is generally better to include
	this potential failure mode.}

\section{Failure Mode and Effect Analysis}
The next pages will show the full failure mode and effect analysis (FMEA) for REVITALIZE:
\begin{landscape}
	\begin{figure}[ht]
		\centering
		\includegraphics[angle=360, scale=1.42]{FMEAPart1.png}
		\caption{Part 1 of FMEA}
	\end{figure}
\end{landscape}

\section{Safety and Security Requirements}

\wss{Newly discovered requirements.  These should also be added to the SRS.  (A
	rationale design process how and why to fake it.)}

\section{Roadmap}

\wss{Which safety requirements will be implemented as part of the capstone timeline?
	Which requirements will be implemented in the future?}

\end{document}